\section{Introduction}

%\begin{itemize}
%\item Importance of BX, BX is a solution of view update problem in database.
%\item goodness of 
%\item Explanation of put-based BX: BiGUL.
%\item Current status of BiGUL: Fastest BX language in the world
%\item But there is a problem: Efficiency of compose evaluation. The current implementation of BiGUL does not save the intermediate states, the number of get is quadratic. This is not good.
%\item To solve this problem we use an idea: introduce pg : combination of put and get. Then, no information will be lost in a specific condition.
%  an idea from reversible computation: not to lose any information.
%\item We extend pg with several ideas to produce faster implementation.
%\item 
%\end{itemize}

%\begin{itemize}
%\item Importance of BX, BX is a solution of view update problem in database.
%\item goodness of 
%\item Explanation of put-based BX: BiGUL.
%\item Current status of BiGUL: Fastest BX language in the world
%\item But there is a problem: Efficiency of compose evaluation. The current implementation of BiGUL does not save the intermediate states, the number of get is quadratic. This is not good.
%\item To solve this problem we use an idea: introduce pg : combination of put and get. Then, no information will be lost in a specific condition.
%  an idea from reversible computation: not to lose any information.
%\item We extend pg with several ideas to produce faster implementation.
%\item 
%\end{itemize}

In software, there are strong demands for synchronizing data. In database community this is known as ``view update problem'' and researched for a long time \cite{viewupdate}.
% A Survey to View Update Problem 
As a solution for this problem, bidirectional transformation (BX) is introduced \cite{}. As an example, let us consider a small BX program of $pHead$\footnote{The actual program is shown in the next section.}. BX provides two functions: $get$ and $put$. The following is one example of $pHead$'s evaluation.

\begin{center}
  \includegraphics[width=5cm]{./temp_fig/bxexample1.eps}
\end{center}
%\end{picture}

Here, the original source \texttt{s}, \texttt{[1,2]}, is given by a user. $get$ is a projection: $get$ of $pHead$ picks the first element of the given original source as a view.
%\texttt{get pHead [1,2] = 1}.
After the view \texttt{v} is obtained, the user can modify the view.
In this case, he or she modified the view to \texttt{8} from \texttt{1}.
From this updated view, how can we obtain new updated source? This is ``view update problem.'' For this, BX provides $put$, update on the original source.
In this example, $put$ of $pHead$ constructs a new list \texttt{[8,2]} from the updated view ($v'$) and the original source ($s$).
%\texttt{put pHead [1,2] 8 = [8,2]}

For more various evaluations, we can use composition of programs.
Let us consider another BX example program, three composition of $pHead$: $pHead \circ pHead \circ pHead$.
The following is one example of its evaluation.


\begin{center}
  \includegraphics[width=10cm]{./temp_fig/bxexample2.eps}
\end{center}

In this example, the original source \texttt{s} is \texttt{[[[1,2],[3]],[4]]}.
By repetitious evaluation of $get$s of $pHead$, we can obtain the view $v$, \texttt{1}. To obtain the final updated source \texttt{s'}, we need to evaluate $put$ three times. The first put evaluation is from \texttt{i2} and \texttt{v'} and we can obtain \texttt{i2'}.

For evaluating put-direction, there are two strategies. One is ``not keeping any intermediate states and obtaining them by evaluation when they are needed.'' In this example, ``intermediate states'' are \texttt{i1} and \texttt{i2}.
If we choose this strategy, the number of $get$s will be quadratic. This is a problem, because the evaluation will be slow if the BX programs include many compositions. In this example, two $get$s are required for obtaining \texttt{i2} and one $get$ is required for obtaining \texttt{i1}. In total, three $get$s are required.

The other strategy is ``keeping all intermediate states and using them when they are needed.'' This strategy causes a storage problem. If views includes most information, for evaluating $put$ the needed parts of the original source will be small. It is redundant to keep all intermediate states.

In this work, we choose another strategy ``keeping complements and using them when they are needed.'' Complements are smaller program fragments than the original intermediate states. Therefore this strategy solves the previous two problems.
Readers might already know that it is a hard problem to obtain complements \cite{}. In our work, thanks to the following finding it is not hard:

\vspace{2mm}
In very-well behaved BX programs, 
$put$ can be a complement function for $get$, $get$ can be a complement function for $put$.
\vspace{2mm}

For using complement, we combine $put$ and $get$, and produce new function $pg$ in Section 3.
This is a kind of tupling. Because parts of $put$ and $get$ are doing the same computation, we can make efficient by tupling them.

As an example, let us consider the previous example again: $pHead \circ pHead \circ pHead$.


\begin{center}
  \includegraphics[width=9cm]{./temp_fig/bxexample3.eps}
\end{center}


In the figure, \texttt{c1} and \texttt{c2} are complements. The points in this figure are two. After evaluation of the first $pg$, we do not need to keep the original source \texttt{s}, because all its information is in \texttt{c1} and \texttt{i1}. 
This evaluation looks better than previous two strategies: this does not require repeated evaluation and require smaller storage than the original sources. 
Actually, the simple combined $pg$ is not effective for left associative compositions. To achieve efficient evaluation, we introduce $cpg$, lazy update version of $pg$, in Section 4 and $kpg$, lazy evaluation version of $cpg$, in Section 5. In Section 6 we combine $pg$ and $kpg$ to achieve efficiency for both associative compositions. In Section 7 we compare all approaches including the original BX approach by the evaluation time and memory allocation. 


%There are two types of bidirectional transformation. One is get-based bidirectional transformation \cite{} that receives the definition of $get$ and provides $put$. A problem of this approach is that it is possible to derive many correct $put$s. To choose one put, hulistics and programmer's intentions are needed. Another approach is put-based bidirectional transformation \cite{} that received the definition of $put$ and provide $get$. From a $put$ we can derive an unique $get$. 

%BiGUL \cite{} is one of put-based bidirectional transformation language. It is fastest bidirectional language in the world now. However, it contains a problem when the BiGUL programs including many compositions. BiGUL does not store the intermediate states when $get$s are re-evaluated so many times. In this case the number of get is quadratic.

%\vspace{5mm}

%Figure that shows many gets re-evaluation 

%\vspace{5mm}

% general, concrete problem
% get-based also have the same problem. even in lenses. more general problem in
% The example should be from lenses.
% traversal, not exactly the same computation

%To solve this problem and make BiGUL evaluation efficient, we use two techniques. The first technique is tupling, well-known technique for efficiency \cite{}. Its basic idea is to reduce the common computations by combining two evaluations. In our work, we combines the evaluation of $put$ and $get$, and introduce $pg$. Thanks to tupling, we do not lose information like intermediate states.
%The second technique is lazy update and lazy computation. $pg$ itself includes some inefficiency. For removing the inefficiency we do the update and computation lazily.

%The rest of this paper is the following. We introduce a subset of BiGUL, miniBiGUL, and show its syntax, semantics, and properties in Section 2. Combining the semantics of put and get in Section 2, we construct $pg$ and show that it is self-inverse in Section 3.
%We apply three optimizations to $pg$ in sequence. The first one is keeping update of BX lazily, and we obtain $cpg$ in Section 4. The second one is lazy evaluation to avoid redundant computations, and we obtain $kpg$ in Section 5. The third one is combination of $pg$ and $kpg$ in Section 6, because they are strong for different type of programs. In Section 7, we show experiemntal results of $pg$, $cpg$, $kpg$, $xpg$ and miniBiGUL.
%In Section 8 we discuss about related work, and conclude in Section 9.


% Story ->
% reversible is needed not to lose any information -> pg is introduced.
% pg is not good for lassoc -> cpg (xpg_2) is introduced.
% cpg can be improved -> kpg(xpg) is introduced.
% kpg is slower than pg for rassoc. combined version -> xpg.

% Evaluation result: Comparison with OCaml implementation of minBiGUL

%\begin{itemize}
%\item Importance of BX, BX is a solution of view update problem in database.
%\item Explanation of put-based BX: BiGUL.
%\item Current status of BiGUL: Fastest BX language in the world
%\item But there is a problem: Efficiency of compose evaluation. The current implementation of BiGUL does not save the intermediate states, the number of get is quadratic. This is not good.
%\item To solve this problem we use an idea: introduce pg : combination of put and get. Then, no information will be lost in a specific condition.
%  an idea from reversible computation: not to lose any information.
%\item We extend pg with several ideas to produce faster implementation.
%\item 
%\end{itemize}

% get & put independently, but for efficient computation
% we need to tuple them to help composition
% 2 or 3 examples to show tupling solves the problem.
% other examples are for showing kpg and cpg is needed.

\subsubsection*{Contribution of this paper}

We can summarize contribution of this paper as follows.

\begin{itemize}
\item Improvement of evaluation efficiency
  \begin{itemize}
  \item This is both in evaluation time and memory allocation
  \end{itemize}
\item Optimization by tupling
  \begin{itemize}
  \item In BX, as far as authors know optimization by $get$ and $put$ more tight is the first attempt
  \end{itemize}
\end{itemize}

%\begin{itemize}
%\item Improvement of evaluation efficiency
%  \begin{itemize}
%  \item Quadratic to exponential? 
%  \item The introduced extentions (maybe) use more memory than the original approach, but they are faster than the original.
%  \end{itemize}
%\item Get and put more tight

%\item (Tupling introduces some constructions)
%  \begin{itemize}
%  \item A pair of put and get is self-inverse (involution).
%  \end{itemize}
%\item careful lazyness is needed in our implementation for efficiency
%  lazy update \& lazy computation are also good optimization for BX.
%\item Reveal a part of relationship between BX and Reversible.
%  \begin{itemize}
%  \item Very well-behaved (no adaptive-case) \& The same evaluation path $\to$ No garbage Reversible Computation.
%  \item 
%  \item If we can obtain good parameter (dummy) for a BiGUL program,
%    put(s, dummy) can be a complement function.
%  \end{itemize}

%\subsection{Construction of this paper}
