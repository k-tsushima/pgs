\section{Conclusion and Future Work}

In this paper we focus on efficiency of composition of BX programs.
The essential finding comes from the idea of tupling: in very-well behaved BX programs we can use $put$ as a compliment function for $get$, and vice versa.
Based on the idea, we introduced $pg$, and improved it by several optimization techniques. From experimental results, our fastest approach $xpg$ is faster than other approaches for non purely right associative compositions. For right associative compositions, the original approach (miniBiGUL) is faster than $xpg$ because $xpg$ has some overhead cost. However this is not a problem, because usually programs are mix of left and right associative. If programmers know that their programs are purely right associative, they can choose miniBiGUL.

We will continue our work on the following points. First, our target language is limited to very-well behaved, because our main idea requires the put-put property. However, for practical programs, we need to extend our work to overcome the current limitations.

\begin{itemize}
\item Extend our approach -- overcome our limitations
  \begin{itemize}
  \item treat well-behaved programs -- How to treat adaptive cases
  \item In case expressions, the programs that use the different paths (put and get)
  \end{itemize}
\item How to obtain dummy?
\item Type system \& Typing -- for safety
\end{itemize}
