\section{Related Work} \label{sect:related}

%\subsubsection*{BX Languages and Implementations}

Since the pioneering work of lens \cite{Foster:2007:CBT:1232420.1232424}, many BX languages have been proposed \cite{Bohannon06relationallenses:, Bohannon:2008:BRL:1328438.1328487,Buchmann:2018:BFI:3362232.3362263,Ko:2016:BFV:2847538.2847544,LeblebiciAS14,Samimi-Dehkordi18,Cicchetti2011,Hinkel:2019:CPB:3318595.3318634}.
%(memo: BXtend, eMoflon, EVL+Strace, JTL, NMF. SDMLib should be added?)
%based on several application scenarios and BX lenses works \cite{}.
Although much progress has been made on the semantics and correctness of BX programs for the past years, as far as we are aware, little work has been done on optimization of BX programs \cite{Horn:2018:IRL:3243631.3236769}. Anjorin et. al. introduces the first benchmark for BX languages and compared them \cite{Anjorin2019}, but a systematic improvement for practical implementation of BX languages is still missing. This paper shows the first attempt of improving efficiency of BX composition evaluation.

The baseline of this work is the BX language BiGUL \cite{Ko:2016:BFV:2847538.2847544, Ko:2017:ABB:3177123.3158129}, and we compare BiGUL's method (in Section~\ref{sect:minbigul}) with our methods (in Sections~\ref{sect:minbigulm} and \ref{sect:xpg}). From experimental results of left associative BX composition programs, we can see that our memoization methods are faster than the original BiGUL's evaluation method.
While we focus on BiGUL, our methods are general and should be applicable to other BX languages.

%find that there is no best approach for all BX programs. It is important to choose good method depending on the BX programs and their inputs.
% its implementation uses ``not keeping any intermediate states and obtaining them by evaluation when they are needed'' strategy. Other BX languages might have the same problem, and our approach can be applicable.

%\subsubsection*{Optimization Techniques}

Our work is related to many known optimization methods for unidirectional programs. Memoization \cite{Bellman:2003:DP:862270,MICHIE1968} is a technique to avoid repeated redundant computation. In our case, we show that two specific memoization methods can be used for bidirectional programs.
To deal with inefficiency due to compositions, many fusion methods have been studied \cite{Wadler:1988:DTP:80099.80104} to merge a composition of two (recursive) programs into one. However, under the context of bidirectional programs, we need to consider not only compositions of recursive programs but also compositions inside a recursive program (as we have seen in \mbox{\it bfoldr}). This paper focuses on the composition inside a recursion, where compositions are produced dynamically at runtime. We tackled the problem by using tupling \cite{Fokkinga90}, lazy update and lazy computation \cite{Henderson:1976:LE:800168.811543, Hudak:2007:HHL:1238844.1238856}.


%In this paper, we use several well-known techniques: tupling, lazy update, and lazy computation (\textcolor{red}{also fusion?}).

%Tupling \cite{tupling} is an optimization technique by combining several computations. Although tupling is usually used for reducing the evaluation cost, we use it to reduce the size of the things need to keep.

%We introduce lazy update to remove repeated evaluation of put. This is different from lazy evaluation. In lazy evaluation we have everything that is needed to evaluate. In our lazy update, because view information is missing, we postpone the evaluation. (\textcolor{red}{Relation between delta lenses?})

%Lazy evaluation is well-known and widly-used optimization technique to reduce the redundant evaluation. Because we use OCaml for implementation, we need to introduce this explicitly. If we use Haskell, a lazy evaluated language, we do not need to introduce them.

%\begin{itemize}
    %\item BiGUL papers (PEPM, POPL, tutorial?)
    %\item the first BX paper
    %\item get-based BX
    %\item put-based BX?
%    \item implementation of BX, they might have the same problem, this paper will be first attempt, semantics and correctness is considered more, but efficiency is  lacked.
%    \item optimization techniques : tupling, lazy computation, fusion transformation?
%    \item (view update problem paper)
%\end{itemize}
