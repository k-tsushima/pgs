\section{Bidirectional Programming Language: minBiGUL} \label{sect:minbigul}

minBiGUL, our target language in this paper, is a subset of BiGUL which is a simple yet powerful putback-based bidirectional language. BiGUL supports two transformations: a forward transformation $get$ producing a view from a source and a backward transformation $put$ taking a source and a modified view to produce an updated source. Intuitively, if we have a BiGUL program $bx$, these two transformations are following functions:

    $\qtab \getbxinline{bx} : s \to v, \quad \putbxinline{bx} : s * v \to v$

BiGUL is well-behaved \cite{Pacheco:2014:MCP:2543728.2543737} since two functions $\putbxinline{bx}$ and $\getbxinline{bx}$ satisfy the round-trip laws as follows:
    
    $\qtab \putbx{bx}{s}{(\getbx{bx}{s})} = s \qquad [\textsc{GetPut}]$
    
    $\qtab \getbx{bx}{(\putbx{bx}{s}{v})} = v \qquad [\textsc{PutGet}]$

The \textsc{GetPut} law means that if there is no change to the view, there should be no change to the source. The \textsc{PutGet} law means that we can recover the modified view by applying the forward transformation to the updated source.

minBiGUL inherited from BiGUL also supports transformations $put$ and $get$ which satisfy two above laws. In addition, we restrict adaptive cases of BiGUL on minBiGUL. Then $put$ and $get$ satisfy one more law, \textsc{PutPut} \cite{Foster:2007:CBT:1232420.1232424}, like the following:

    $\qtab \putbx{bx}{(\putbx{bx}{s}{v'})}{v} = \putbx{bx}{s}{v} \qquad [\textsc{PutPut}]$

The \textsc{PutPut} law means that a source update should overwrite the effect of previous source updates. Due to the satisfaction of three laws, \textsc{GetPut}, \textsc{PutGet} and \textsc{PutPut}, minBiGUL is very well-behaved.

\subsection{Syntax}

The syntax of minBiGUL is briefly written as follows:

    $bx := Skip \ h \
        | \ Replace \
        | \ Prod \ bx_1 \ bx_2 \
        | \ RearrS \ f_1 \ f_2 \ bx \
        | \ RearrV \ g_1 \ g_2 \ bx \\
    \qtabs{2} | \ Case \ cond_{sv} \ cond_{s} \ bx_1 \ bx_2 \
        | \ Compose \ bx_1 \ bx_2$

A minBiGUL program may be either a skip of a function or a replacement or a product of two minBiGUL programs or a source/view rearrangement or a case combinator without adaptive cases or a composition of some minBiGUL programs. We use numbers, pairs and lists to construct the program input including the source and/or the view.

For source/view rearrangement, BiGUL uses just one lambda expression to express how to deconstruct as well as reconstruct data. It is a kind of bijection. However, to be able to implement it in OCaml, the environment used for developing minBiGUL and solutions in the paper, we need give two functions which one is the inverse of the other. In the above syntax, $f_2 = f_1^{-1}$ and $g_2 = g_1^{-1}$.

To help make demonstration more direct, we provide the following alternatives representation: $Prod \ bx_1 \ bx_2 \equiv bx_1 \times bx_2, \ Compose \ bx_1 \ bx_2 \equiv bx_1 \circ bx_2$. In general, $\circ$ has a higher priority than $\times$. Their associativity precedence can be either left or right or mixture, but are not set by default. We need to explicitly write programs that use these operators.

\subsection{Semantics}

\begin{multicols}{2}
    \begin{definition} \label{def:minbigulput}
        $\putbx{bx}{s}{v}$

        \noindent $\putbx{Skip \ h}{s}{v} = \\
            \tab \sif{h \ s = v}{s}{\text{fail}}$
    
        \noindent $\putbx{Replace}{s}{v} = v$
    
        \noindent $\putbx{\product{bx_1}{bx_2}}{(s_1, s_2)}{(v_1, v_2)} = \\
            \tab ((\putbx{bx_1}{s_1}{v_1}),(\putbx{bx_2}{s_2}{v_2}))$
    
        \noindent $\putbx{\rearrs{f_1}{f_2}{bx}}{s}{v} = \\
            \tab f_2 \ (\putbx{bx}{(f_1 \ s)}{v})$
    
        \noindent $\putbx{\rearrv{g_1}{g_2}{bx}}{s}{v} = \\
            \tab \putbx{bx}{s}{(g_1 \ v)}$
    
        \noindent $\putbx{\casebx{cond_{sv}}{cond_{s}}{bx_1}{bx_2}}{s}{v} =\\
            \tab \text{if} \ {cond_{sv} \ s \ v} \\
            \tab \text{then} \ s' \Leftarrow \putbx{bx_1}{s}{v} \\
            \tab \text{else} \ s' \Leftarrow \putbx{bx_2}{s}{v} \\
            \tab \text{fi} \ cond_{s} \ s'; \ \text{return} \ s'$
    
        \noindent $\putbx{bx_1 \circ bx_2}{s}{v} = \\
            \tab \putbx{bx_1}{s}{(\putbx{bx_2}{(\getbx{bx_1}{s})}{v})}$
    \end{definition}
\columnbreak
    \begin{definition} \label{def:minbigulget}
        $\getbx{bx}{s}$

        \noindent $\getbx{Skip \ h}{s} = \\ 
            \tab h \ s$

        \noindent $\getbx{Replace}{s} = s$

        \noindent $\getbx{\product{bx_1}{bx_2}}{(s_1,s_2)} = \\
            \tab ((\getbx{bx_1}{s_1}),(\getbx{bx_2}{s_2}))$

        \noindent $\getbx{\rearrs{f_1}{f_2}{bx}}{s} = \\ 
            \tab \getbx{bx}{(f_1 \ s)}$

        \noindent $\getbx{\rearrv{g_1}{g_2}{bx}}{s} = \\ 
            \tab g_2 \ (\getbx{bx}{s})$

        \noindent $\getbx{\casebx{cond_{sv}}{cond_{s}}{bx_1}{bx_2}}{s} = \\
            \tab \text{if} \ {cond_{s} \ s} \\
            \tab \text{then} \ v' \Leftarrow \getbx{bx_1}{s} \\ 
            \tab \text{else} \ v' \Leftarrow \getbx{bx_2}{s} \\ 
            \tab \text{fi} \ {cond_{sv} \ s \ v'}; \ \text{return} \ v'$

        \noindent $\getbx{bx_1 \circ bx_2}{s} = \\ 
            \tab \getbx{bx_2}{(\getbx{bx_1}{s})}$
    \end{definition}
\end{multicols}

We define the semantics of $put$ and $get$ as in Definitions~\ref{def:minbigulput} and \ref{def:minbigulget} respectively. Instead of using $v'$ in the $put$ direction like Figures~\ref{fig:eval-phead}, \ref{fig:eval-comp-phead} and \ref{fig:eval-comp-phead-2}, we just simply use notation $v$ as the updated view. The later definitions also follow this convention.

In the above definitions, we use if-then-else-fi statements to express semantics of $\putbxinline{Case}$ and $\getbxinline{Case}$ where $\Leftarrow$ means assignments. This statement is useful to describe many functions related to $Case$ in this paper. Statement (if $E_1$ then $X_1$ else $X_2$ fi $E_2$) means if the test $E_1$ is true, the statement $X_1$ is executed and the assertion $E_2$ must be true, otherwise, i.e. $E_2$ is false, the statement $X_2$ is executed and the assertion $E_2$ must be false. If the values of $E_1$ and $E_2$ are distinct, the if-then-else-fi structure is undefined. We can write the equivalent if-then-else statement as follows:

$\tab \text{if } E_1 \text{ then } X_1 \text{ else } X_2 \text{ fi } E_2; S\\
\tabs{2} \ \equiv \text{if }E_1 = true \text{ then } \{ X_1; \text{ if } E_2 = true \text{ then } S \text{ else assert } false; \}\\
    \tabs{4} \text{else } \{ X_2; \text{ if } E_2 = false \text{ then } S \text{ else assert } false; \}$

\noindent We now consider the definition of $phead$ in minBiGUL:\\
$\tab phead = RearrS \ f_1 \ f_2 \ bx_s \ \text{where: } f_1 = \lambda (s::ss).(s,ss), \, f_2 = \lambda (s,ss).(s::ss),\\
    \qtab bx_s = RearrV \ g_1 \ g_2 \ bx_v \ \text{where: } g_1 = \lambda v.(v,()), \, g_2 = \lambda (v,()).v,\\
        \qtabs{2} bx_v = \product{Replace}{(Skip \ (\lambda \_.())}$

The above program rearranges the source, a non-empty list, to a pair of its head element $s$ and its tail $ss$, and the view to a pair $(v, ())$, then we can use $v$ to replace $s$ and $()$ to keep $ss$. Intuitively, $\putbx{phead}{s_0}{v_0}$ returns a list whose head is $v_0$ and tail is the tail of $s_0$, and $\getbx{phead}{s_0}$ returns the head of the list $s_0$. For instance, $\putbx{phead}{[1,2,3]}{100} = [100,2,3]$ and $\getbx{phead}{[1,2,3]} = 1$. If we wanna update the head element of the head element of a list of lists by using the view, we can define a composition like $phead \circ phead$. For example:

$\tab \putbx{phead \circ phead}{[[1,2,3],[\,],[4,5]]}{100} = [[100,2,3],[\,],[4,5]]$

$\tab \getbx{phead \circ phead}{[[1,2,3],[\,],[4,5]]} = 1$

In the end of this section, let's see the definition of $bfoldr$ which is a putback function of an important higher-order function on lists, $foldr$:

$\tab bfoldr \ bx = Case \ cond_{sv} \ cond_s \ bx_1 \ bx_2$ where:\\
$\tabs{3} \ cond_{sv} = \lambda (s_1,s_2).\lambda v.(s_1 = [\ ]), \ cond_s = \lambda (s_1,s_2).(s_1 = [\ ])$\\
$\tabs{3} \ bx_1 = RearrV \ g_1 \ g_2 \ bx_v$ where: \\
    $\tabs{4} g_1 = g_2^{-1} = \lambda [v].(v,[\ ]), bx_v = (Skip \ (\lambda \_.())) \times Replace$\\
$\tabs{3} \ bx_2 = RearrS \ f_1 \ f_2 \ bx_s$ where:\\
    $\tabs{4} f_1 = f_2^{-1} = \lambda ((x:xs),e).(x,(xs,e))$, 
    $bx_s = ((Replace \times bfoldr \ bx) \circ bx)$

\noindent $bfoldr$ will look like a composition of many programs at some points in the evaluation if we slow down this process. Since $\circ$ has a higher priority than $\times$, it is seemingly impossible to transform $bfoldr$ from the left associative composition style to the right one. 
