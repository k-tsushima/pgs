\section{Bidirectional Programming Language: minBiGUL} \label{sect:minbigul}

The target language in this paper, minBiGUL, is a very-well behaved subset of BiGUL, which is a simple, yet powerful putback-based bidirectional language. 

BiGUL supports two transformations: a forward transformation $get$ producing a view from a source and a backward transformation $put$ taking a source and a modified view to produce an updated source. Intuitively, if we have a BiGUL program $bx$, these two transformations are the following functions:

 \smallvspace
    $\qtab \getbxinline{bx} : s \to v, \quad \putbxinline{bx} : s * v \to v$
 \smallvspace
 
BiGUL is well-behaved \cite{Pacheco:2014:MCP:2543728.2543737} since two functions $\putbxinline{bx}$ and $\getbxinline{bx}$ satisfy the round-trip laws as follows:

\smallvspace
    $\qtab \putbx{bx}{s}{(\getbx{bx}{s})} = s \qquad [\textsc{GetPut}]$
    
    $\qtab \getbx{bx}{(\putbx{bx}{s}{v})} = v \qquad [\textsc{PutGet}]$
\smallvspace

\noindent The \textsc{GetPut} law means that if there is no change to the view, there should be no change to the source. The \textsc{PutGet} law means that we can recover the modified view by applying the forward transformation to the updated source.

minBiGUL inherits from BiGUL both 
% supports 
transformations, $put$ and $get$, which satisfy the two laws above. 
% In addition, 
Because we restrict the `adaptive case' of BiGUL in minBiGUL,
% In this case, 
$put$ and $get$ satisfy one more law, namely the \textsc{PutPut} law~\cite{Foster:2007:CBT:1232420.1232424}:
% like the following:

\smallvspace
$\qtab \putbx{bx}{(\putbx{bx}{s}{v'})}{v} = \putbx{bx}{s}{v} \qquad [\textsc{PutPut}]$
\smallvspace

\noindent The \textsc{PutPut} law means that a source update should overwrite the effect of previous source updates. 
Because minBiGUL satisfies all three laws, \textsc{GetPut}, \textsc{PutGet} and \textsc{PutPut}, it is very well-behaved~\cite{Foster:2007:CBT:1232420.1232424}.
% Due to the satisfaction of three laws, \textsc{GetPut}, \textsc{PutGet} and \textsc{PutPut}, minBiGUL is very well-behaved \cite{Foster:2007:CBT:1232420.1232424}.

\subsection{Syntax}

The syntax of minBiGUL is briefly written as follows:

\smallvspace
    $bx ::= Skip \ h \
        | \ Replace \
        | \ Prod \ bx_1 \ bx_2 \
        | \ RearrS \ f_1 \ f_2 \ bx \
        | \ RearrV \ g_1 \ g_2 \ bx \\
    \qtabs{2} | \ Case \ cond_{sv} \ cond_{s} \ bx_1 \ bx_2 \
        | \ Compose \ bx_1 \ bx_2$
\smallvspace
        
A minBiGUL program is either a skip of a function, a replacement, a product of two 
%minBiGUL 
programs, a source/view rearrangement, a case combinator (without adaptive cases), or a composition of 
% some minBiGUL 
two programs. We use numbers, pairs and lists to construct the program inputs including the source and/or the view.

For source/view rearrangement, BiGUL uses 
% just one 
a lambda expression to express how to deconstruct as well as reconstruct data. It is a kind of bijection. However, to be able to implement it in OCaml, the environment used for developing minBiGUL and solutions in the paper, we need to require two functions which one is the inverse of the other. In the above syntax, $f_2 = f_1^{-1}$ and $g_2 = g_1^{-1}$.

To help make demonstration more direct, we provide the following alternatives representation: $Prod \ bx_1 \ bx_2 \equiv bx_1 \times bx_2, \ Compose \ bx_1 \ bx_2 \equiv bx_1 \circ bx_2$. In general, $\circ$ has a higher priority than $\times$. Their associativity precedence can be either left or right or mixture, but are not set by default. We need to explicitly write programs that use these operators.

\subsection{Semantics}

The semantics of $put$ and $get$ is shown in Definitions~\ref{def:minbigulput} and \ref{def:minbigulget}, respectively. Instead of using the name $v'$ for the updated view in the $put$ direction, like Figures~\ref{fig:eval-phead}, \ref{fig:eval-comp-phead} and \ref{fig:eval-comp-phead-2}, we 
%just
simply 
%use notation 
use~$v$ below. The later definitions also follow this convention.

\begin{multicols}{2}
    \begin{definition} \label{def:minbigulput}
        $\putbx{bx}{s}{v}$

        \noindent $\putbx{Skip \ h}{s}{v} = \\
            \tab \sif{h \ s = v}{s}{\text{undefined}}$
    
        \noindent $\putbx{Replace}{s}{v} = v$
    
        \noindent $\putbx{\product{bx_1}{bx_2}}{(s_1, s_2)}{(v_1, v_2)} = \\
            \tab ((\putbx{bx_1}{s_1}{v_1}),(\putbx{bx_2}{s_2}{v_2}))$
    
        \noindent $\putbx{\rearrs{f_1}{f_2}{bx}}{s}{v} = \\
            \tab f_2 \ (\putbx{bx}{(f_1 \ s)}{v})$
    
        \noindent $\putbx{\rearrv{g_1}{g_2}{bx}}{s}{v} = \\
            \tab \putbx{bx}{s}{(g_1 \ v)}$
    
        \noindent $\putbx{\casebx{cond_{sv}}{cond_{s}}{bx_1}{bx_2}}{s}{v} =\\
            \tab \textnormal{if} \ {cond_{sv} \ s \ v} \\
            \tab \textnormal{then} \ s' \Leftarrow \putbx{bx_1}{s}{v} \\
            \tab \textnormal{else} \ s' \Leftarrow \putbx{bx_2}{s}{v} \\
            \tab \textnormal{fi} \ cond_{s} \ s'; \ \textnormal{return} \ s'$
    
        \noindent $\putbx{bx_1 \circ bx_2}{s}{v} = \\
            \tab \putbx{bx_1}{s}{(\putbx{bx_2}{(\getbx{bx_1}{s})}{v})}$
    \end{definition}
\columnbreak
    \begin{definition} \label{def:minbigulget}
        $\getbx{bx}{s}$

        \noindent $\getbx{Skip \ h}{s} = \\ 
            \tab h \ s$

        \noindent $\getbx{Replace}{s} = s$

        \noindent $\getbx{\product{bx_1}{bx_2}}{(s_1,s_2)} = \\
            \tab ((\getbx{bx_1}{s_1}),(\getbx{bx_2}{s_2}))$

        \noindent $\getbx{\rearrs{f_1}{f_2}{bx}}{s} = \\ 
            \tab \getbx{bx}{(f_1 \ s)}$

        \noindent $\getbx{\rearrv{g_1}{g_2}{bx}}{s} = \\ 
            \tab g_2 \ (\getbx{bx}{s})$

        \noindent $\getbx{\casebx{cond_{sv}}{cond_{s}}{bx_1}{bx_2}}{s} = \\
            \tab \textnormal{if} \ {cond_{s} \ s} \\
            \tab \textnormal{then} \ v' \Leftarrow \getbx{bx_1}{s} \\ 
            \tab \textnormal{else} \ v' \Leftarrow \getbx{bx_2}{s} \\ 
            \tab \textnormal{fi} \ {cond_{sv} \ s \ v'}; \ \textnormal{return} \ v'$

        \noindent $\getbx{bx_1 \circ bx_2}{s} = \\ 
            \tab \getbx{bx_2}{(\getbx{bx_1}{s})}$
    \end{definition}
\end{multicols}

The two definitions use if-then-else-fi statements 
%express 
to define the semantics of $\putbxinline{Case}$ and $\getbxinline{Case}$, where $\Leftarrow$ denotes an assignment. This statement is useful to describe many functions related to $Case$ in this paper. Statement (if $E_1$ then $X_1$ else $X_2$ fi $E_2$) means ``if the test $E_1$ is true, the statement $X_1$ is executed and the assertion $E_2$ must be true, otherwise, if $E_1$ is false, the statement $X_2$ is executed and the assertion $E_2$ must be false.'' If the values of $E_1$ and $E_2$ are distinct, the if-then-else-fi structure is undefined. We can write the equivalent if-then-else statement as follows:

\smallvspace
$\tab \text{if } E_1 \text{ then } X_1 \text{ else } X_2 \text{ fi } E_2; S\\
\tabs{2} \ \equiv \text{if }E_1 = \textit{true} \text{ then } \{ X_1; \text{ if } E_2 = true \text{ then } S \text{ else } \textit{undefined}\}\\
    \tabs{4} \text{else } \{ X_2; \text{ if } E_2 = \textit{false} \text{ then } S \text{ else } \textit{undefined} \}$
\smallvspace
    
\noindent As an example of minBiGUL program, consider the definition of $phead$ :\\
\smallvspace
$\tab phead = RearrS \ f_1 \ f_2 \ bx_s \ \text{where: } f_1 = \lambda (s::ss).(s,ss), \, f_2 = \lambda (s,ss).(s::ss),\\
    \qtab bx_s = RearrV \ g_1 \ g_2 \ bx_v \ \text{where: } g_1 = \lambda v.(v,()), \, g_2 = \lambda (v,()).v,\\
        \qtabs{2} bx_v = \product{Replace}{(Skip \ (\lambda \_.())}$
\smallvspace
        
The above program rearranges the source, a non-empty list, to a pair of its head element $s$ and its tail $ss$, and the view to a pair $(v, ())$, then we can use $v$ to replace $s$ and $()$ to keep $ss$. Intuitively, $\putbx{phead}{s_0}{v_0}$ returns a list whose head is $v_0$ and tail is the tail of $s_0$, and $\getbx{phead}{s_0}$ returns the head of the list $s_0$. For instance, $\putbx{phead}{[1,2,3]}{100} = [100,2,3]$ and $\getbx{phead}{[1,2,3]} = 1$. If we want to update the head element of the head element of a list of lists by using the view, we can define a composition like $phead \circ phead$. For example:

\smallvspace
$\tab \putbx{phead \circ phead}{[[1,2,3],[\,],[4,5]]}{100} = [[100,2,3],[\,],[4,5]]$

$\tab \getbx{phead \circ phead}{[[1,2,3],[\,],[4,5]]} = 1$
\smallvspace

To close this section,
% In the end of this section, 
let us see the minBiGUL definition of $\mathit{bfoldr}$ which is a putback function of an important higher-order function on lists, $\mathit{foldr}$:

\smallvspace
$\tab \mathit{bfoldr} \ bx = Case \ cond_{sv} \ cond_s \ bx_1 \ bx_2$ where:\\
$\tabs{3} \ cond_{sv} = \lambda (s_1,s_2).\lambda v.(s_1 = [\ ]), \ cond_s = \lambda (s_1,s_2).(s_1 = [\ ])$\\
$\tabs{3} \ bx_1 = RearrV \ g_1 \ g_2 \ bx_v$ where: \\
    $\tabs{4} g_1 = g_2^{-1} = \lambda [v].(v,[\ ]), bx_v = (Skip \ (\lambda \_.())) \times Replace$\\
$\tabs{3} \ bx_2 = RearrS \ f_1 \ f_2 \ bx_s$ where:\\
    $\tabs{4} f_1 = f_2^{-1} = \lambda ((x:xs),e).(x,(xs,e))$, 
$bx_s = ((Replace \times bfoldr \ bx) \circ bx)$
\smallvspace

\noindent In $\mathit{bfoldr}$, the composition is inherently left associative, and the number of composition is dynamically determined by the length of the source list. Because~$\circ$ has a higher priority than~$\times$, it is in general not possible to transform $\mathit{bfoldr}$ from the left associative composition style to the right one. 
%will look like a composition of many programs at some points in the evaluation if we slow down this process. Since $\circ$ has a higher priority than $\times$, it is seemingly impossible to transform $bfoldr$ from the left associative composition style to the right one. 
