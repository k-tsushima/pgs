\section{Memorizing: minBiGUL$_m$}

For minBiGUL or even BiGUL, when evaluating a composition of many programs, $get$s are reevaluated so many times since no intermediate state is stored during the evaluation. To avoid reevaluating $get$s, we can save intermediate states in a key-value table where the key is a pair of the program $bx$ and the source $s$, and the value is the result of $\getbx{bx}{s}$. Then, searching an evaluated $get$s on the table is used instead of recomputing it. We define a memory version, minBiGUL$_m$, including $get_m$ and $put_m$ as in definition 3 and 4 respectively. Another function introduced in definition 5, $get_{mh}$, is used to retrieve the value of an evaluated $get$ from the table or compute an unevaluated $get$ by using $get_m$ then store the key-value tuple to the table if necessary.

\begin{definition} $\text{Memo version of } put$

    $\putm{bx}{s}{v} = \text{match } bx \text{ with }\\
        \tab \vert \ bx_1 \circ bx_2 \to \putm{bx_1}{s}{(\putm{bx_2}{(\getmh{bx_1}{s})}{v})} \\
        \tab \vert \ \_ \to \text{similar to } put$
\end{definition}

\begin{multicols}{2}
    \begin{definition} $\text{Memo version of } get$

        $\getm{bx}{s} = \text{match } bx \text{ with }\\
            \tab \ \vert \ bx_1 \circ bx_2 \to\\
            \tabs{2} \text{try } (\text{Hashtbl.find} \ table_g \ (bx, s))\\
            \tabs{2} \text{with Not\_found } \to\\
                \tabs{3} v \Leftarrow \getmh{bx_2}{(\getmh{bx_1}{s})}\\
                \tabs{3} \text{Hashtbl.add } \ table_g \ (bx, s) \ v\\
                \tabs{3} v \\
            \tab \ \vert \ \_ \to \text{similar to } get$
    \end{definition}
\columnbreak
    \begin{definition} $\text{Interface for } get_m$

        $\getmh{bx}{s} =\\
            \tab \text{try } (\text{Hashtbl.find} \ table_g \ (bx, s))\\
            \tab \text{with Not\_found } \to\\
                \tabs{2} v \Leftarrow \getm{bx}{s}\\
                \tabs{2} \text{match } bx \text{ with } \\
                \tabs{2} \ \vert \ bx_1 \circ bx_2 \to v \\
                \tabs{2} \ \vert \ \_ \to \text{Hashtbl.add } \ table_g \ (bx, s) \ v\\
                \qtabs{2} \ v$
    \end{definition}
\end{multicols}

The special point is that not all states are saved when evaluating a program, only composition evaluations are treated. When evaluating $\putm{bx_1 \circ bx_2}{s}{v}$, an application of $get_{mh}$, not $get$ or $get_m$, appears to indicate the saving. When evaluating $\getm{bx_1 \circ bx_2}{s}$, we try to search key ($bx,s$), $bx \equiv bx_1 \circ bx_2$, on the table. The matched value can be used for next steps in the evaluation. If there is no such key, the value $v$ will be calculated by using $get_{mh}$. The further key-value insertion is needed because the later computation could require to reevaluate $\getm{bx_1 \circ bx_2}{s}$ one more time.
