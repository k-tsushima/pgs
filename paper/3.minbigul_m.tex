\section{Adding Memoization: minBiGUL$_m$} \label{sect:minbigulm}

When evaluating the composition of several BX programs, 
% (and the same holds for 
% full BiGUL), 
% all BX languages), 
the same $get$s are evaluated repeatedly.
% because no intermediate states are kept during the evaluation. 
This problem was illustrated in Figure~\ref{fig:eval-comp-phead}. To avoid reevaluating $get$s, and as our first approach to avoid this inefficiency, we introduce memoization in the minBiGUL interpreter.
% can save 
To keep it simple, the intermediate state of a composition is saved in a key-value table where the key is a pair of program $bx$ and source $s$, and the value is the result of evaluating $\getbx{bx}{s}$.
% the first time. 
% Then, instead of recomputing the value, the value in the table is used.
Later the value in the table is used instead of recomputing it. 

The memoizing version, minBiGUL$_m$, needs only two modifications: $get_m$ and $put_m$ (Definitions~\ref{def:putm} and~\ref{def:getm}).

\begin{definition} \label{def:putm} $\text{Memoization version of } put$

    \noindent $\putm{bx}{s}{v} = \match{bx}\\
        \tab \vert \ bx_1 \circ bx_2 \to \\
            \tabs{2} \putm{bx_1}{s}{(\putm{bx_2}{(\getm{bx_1}{s})}{v})} \\
        \tab \vert \ \_ \to \text{similar to } put$
\end{definition}

\begin{definition} \label{def:getm} $\text{Memoization version of } get$

    \noindent $\getm{bx}{s} = \match{bx}\\
    \tab \ \vert \ bx_1 \circ bx_2 \to\\
    \tabs{2} \textnormal{try } (\text{Hashtbl.find} \ table_g \ (bx, s))\\
    \tabs{2} \textnormal{with } \text{Not\_found } \to\\
        \tabs{3} i \Leftarrow \getm{bx_1}{s}; \tab \text{Hashtbl.add } \ table_g \ (bx_1, s) \ i;\\
        \tabs{3} v \Leftarrow \getm{bx_2}{i}; \tab \text{Hashtbl.add } \ table_g \ (bx_2, i) \ v;\\
        \tabs{3} \text{Hashtbl.add } \ table_g \ (bx, s) \ v\\
        \tabs{3} v \\
    \tab \ \vert \ \_ \to \text{similar to } get$
\end{definition}

The evaluation of $\putm{bx_1 \circ bx_2}{s}{v}$ includes two recursive calls of $put_m$ and an external call of $get_m$, which is relatively similar to the evaluation of $\putbx{bx_1 \circ bx_2}{s}{v}$. 
Meanwhile, the evaluation of $\getm{bx_1 \circ bx_2}{s}$ does not merely invoke $get_m$ recursively twice. In case that $bx$ is a composition, the key ($bx,s$) needs to be looked up in the table and the corresponding value would be used for the next steps in the evaluation. If there is no such key, the value of the intermediate state $i$ and the value of $\getbx{bx}{s}$ in $v$ will be calculated. These values along with the corresponding keys will also be stored in the table where the interpreter may later 
leverage instead of reevaluating some states.
% Note as a special point 
% It should be noted 
Note in particular 
that the interpreter does not save all states when evaluating a program, only the intermediate states of a composition.
