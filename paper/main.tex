% This is samplepaper.tex, a sample chapter demonstrating the
% LLNCS macro package for Springer Computer Science proceedings;
% Version 2.20 of 2017/10/04
%
\documentclass[runningheads]{llncs}
%
\usepackage{graphicx}
\usepackage{color}
\usepackage{amsmath}
\usepackage{amsfonts}
\usepackage{amssymb}
\usepackage{multirow}
\usepackage{multicol}
\usepackage{tabularx}
\usepackage{tikz}
\usepackage{pgfplots}
\usepackage{pgfplotstable}
\usepackage{lmodern}
\usepackage{pgffor}
\usepackage{schemata}
\usepackage[colorlinks=true,linkcolor=blue,citecolor=blue]{hyperref}

\usepackage[backend=bibtex,doi=false,url=false,style=nature,sorting=none]{biblatex}
\addbibresource{bibliography}

% Used for displaying a sample figure. If possible, figure files should
% be included in EPS format.
%
% If you use the hyperref package, please uncomment the following line
% to display URLs in blue roman font according to Springer's eBook style:
\renewcommand\UrlFont{\color{blue}\rmfamily}
\renewcommand{\footnotesize}{\scriptsize}

\usetikzlibrary{matrix}
\usepgfplotslibrary{groupplots}
\pgfplotsset{compat=newest}

% \setlength{\parindent}{0pt}
% \renewcommand{\indent}{\hspace*{0pt}}

\newcommand{\tab}{\hspace*{3mm}}
\newcommand{\tabs}[1]{\foreach \n in {1,...,#1}{\tab}}
\newcommand{\qtab}{\hspace*{3mm} \ \quad}
\newcommand{\qtabs}[1]{\foreach \n in {1,...,#1}{\qtab}}

\newcommand{\sif}[3]{\textnormal{if} \ #1 \ \textnormal{then} \ #2 \ \textnormal{else} \ #3}
\newcommand{\product}[2]{#1 \times #2}
\newcommand{\tuple}[2]{(#1 :: #2)}
\newcommand{\rearrs}[3]{RearrS \ #1 \ #2 \ #3}
\newcommand{\rearrv}[3]{RearrV \ #1 \ #2 \ #3}
\newcommand{\casebx}[4]{Case \ #1 \ #2 \ #3 \ #4}

\newcommand{\putbx}[3]{put \, [\![#1]\!] \ #2 \ #3}
\newcommand{\putbxinline}[1]{put \, [\![#1]\!]}
\newcommand{\getbx}[2]{get \, [\![#1]\!] \ #2}
\newcommand{\getbxinline}[1]{get \, [\![#1]\!]}

\newcommand{\putm}[3]{put_m \, [\![#1]\!] \ #2 \ #3}
\newcommand{\putminline}[1]{put_m \, [\![#1]\!]}
\newcommand{\getm}[2]{get_m \, [\![#1]\!] \ #2}
\newcommand{\getminline}[1]{get_m \, [\![#1]\!]}

\newcommand{\putrev}[3]{put_{REV} \, [\![#1]\!] \ {#2} \ {#3}}
\newcommand{\getrev}[2]{get_{REV} \, [\![#1]\!] \ {#2}}

\newcommand{\pg}[3]{pg \, [\![#1]\!] (#2, #3)}
\newcommand{\pginline}[1]{pg \, [\![#1]\!]}

\newcommand{\cpg}[5]{cpg [\![#1]\!] (#2, #3, #4, #5)}
\newcommand{\cpginline}[1]{cpg \, [\![#1]\!]}

\newcommand{\kpg}[7]{kpg [\![#1]\!] (#2, #3, #4, #5, #6, #7)}
\newcommand{\kpginline}[1]{kpg \, [\![#1]\!]}

\newcommand{\xpg}[3]{xpg \, [\![#1]\!] (#2, #3)}
\newcommand{\xpginline}[1]{xpg \, [\![#1]\!]}
\newcommand{\smallvspace}{\vspace{1.2mm}}
\newcommand{\minusvspace}{\vspace*{-5.5mm}}
\newcommand{\minusvspacetwo}{\vspace*{-3.5mm}}

\newcommand{\match}[1]{\textnormal{match } #1 \ \textnormal{ with }}

\newcommand{\newcirc}{\tilde{\circ}}

\makeatletter
\newcommand{\AlignFootnote}[1]{%
    \ifmeasuring@
    \else
        \footnote{#1}%
    \fi
}
\makeatother

\newcolumntype{L}[1]{>{\raggedright\arraybackslash}p{#1}}
\newcolumntype{C}[1]{>{\centering\arraybackslash}p{#1}}
\newcolumntype{R}[1]{>{\raggedleft\arraybackslash}p{#1}}

\begin{document}
%
\title{
  % Strategies for efficient composition of bidirectional programs
  % by memoization and lazy updates
  %  tupling
  %  accumulating updates = lazy updates
  %  based on the idea of reversible computing  
  % \thanks{Supported by organization x.}  
  An Efficient Composition of Bidirectional Programs
  by Memoization and Lazy Update
}
%
\titlerunning{An Efficient Composition of BX Programs}
%\titlerunning{Abbreviated paper title}
% If the paper title is too long for the running head, you can set
% an abbreviated paper title here
%


\author{{Kanae Tsushima\inst{1}} \and %\break% \orcidID{0000-0002-3383-3389}
  {Bach Nguyen Trong\inst{1}} \and %\break %\orcidID{0000-0002-6342-6020}
  {Robert Gl\"{u}ck\inst{2}} \and %\break %\orcidID{0000-0001-6990-3935}
  {Zhenjiang Hu\inst{3}}} %\orcidID{0000-0002-9034-205X}

%\author{anonymous authors}

\authorrunning{K. Tsushima et al.}

%\authorrunning{anonymous authors}

% First names are abbreviated in the running head.
% If there are more than two authors, 'et al.' is used.
%

\institute{National Institute of Informatics, Japan \break
  \email{\{k\_tsushima,bach\}@nii.ac.jp} \and
  University of Copenhagen, Denmark \break
  \email{glueck@acm.org} \and
  Peking University, China \break
  \email{huzj@pku.edu.cn}}

%\institute{anonymous}

\pagestyle{empty}


%
\maketitle              % typeset the header of the contribution
%
\begin{abstract}
%The abstract should briefly summarize the contents of the paper in
%15--250 words.
  Bidirectional transformations (BX) are a solution to the view update problem and widely used for synchronizing data. The semantics and correctness of bidirectional programs has been investigated intensively during the past years, but their efficiency and optimization are not yet fully understood. In this paper, as a first step, we study different evaluation methods to optimized their evaluation. We focus on the interpretive evaluation of BX compositions because we found that these compositions are an important cause of redundant computations. 
 For evaluating BX compositions efficiently, we investigate two memoization methods. The first method, minBiGUL$_m$, uses memoization, which improves the runtime of many BX programs by keeping intermediate results for later reuse. A disadvantage is the familiar tradeoff for keeping and searching values in a table.
  When inputs become large, the overhead increases and the effectiveness decreases. To deal with large inputs, we introduce the second method, $xpg$, that uses tupling, lazy update and lazy evaluation as optimizations. Lazy updates delay updates in closures and enables to use them later.
 Both evaluation methods were fully implemented for minBiGUL and experimentally evaluated. Our experiments show the following results: the original method that is used in the language is faster for right associative compositions, minBiGUL$_m$ is faster for not-large inputs and $xpg$ is faster for large inputs.

 % We show experimental results for several BX programs with simple inputs and large inputs. 


  %To achieve fast evaluation for complicated data, we introduced a novel approach. First, we make $put$ and $get$ more tight, based on the idea of tupling. Because simple tupled result includes redundancies for left associative compositions, we apply two optimization techniques: lazy update and lazy evaluation. We compare the approaches including the strategy used in the actual BX language, and show that our optimized approach is faster than the other approaches (?).

\keywords{Bidirectional transformation \and
  Implementation technique \and
  Efficiency \and 
  Optimization \and Tupling.
}
\end{abstract}

\section{Introduction}

%\begin{itemize}
%\item Importance of BX, BX is a solution of view update problem in database.
%\item goodness of
%\item Explanation of put-based BX: BiGUL.
%\item Current status of BiGUL: Fastest BX language in the world
%\item But there is a problem: Efficiency of compose evaluation. The current implementation of BiGUL does not save the intermediate states, the number of get is quadratic. This is not good.
%\item To solve this problem we use an idea: introduce pg : combination of put and get. Then, no information will be lost in a specific condition.
%  an idea from reversible computation: not to lose any information.
%\item We extend pg with several ideas to produce faster implementation.
%\item
%\end{itemize}

%\begin{itemize}
%\item Importance of BX, BX is a solution of view update problem in database.
%\item goodness of
%\item Explanation of put-based BX: BiGUL.
%\item Current status of BiGUL: Fastest BX language in the world
%\item But there is a problem: Efficiency of compose evaluation. The current implementation of BiGUL does not save the intermediate states, the number of get is quadratic. This is not good.
%\item To solve this problem we use an idea: introduce pg : combination of put and get. Then, no information will be lost in a specific condition.
%  an idea from reversible computation: not to lose any information.
%\item We extend pg with several ideas to produce faster implementation.
%\item
%\end{itemize}

%In software, there are strong demands for synchronizing data. In database community this is known as ``the view update problem'' and researched for a long time.
% A Survey to View Update Problem
%As a solution for this problem, bidirectional transformation (BX) is introduced.
The synchronization of data is a common problem. In the database community this problem is known as “the view update problem” and has been investigated for a long time~\cite{Bancilhon:1981:USR:319628.319634}. Bidirectional transformation (BX) provides a systematic approach to solving this problem.
Consider a small BX program of $phead$\footnote{The actual program is shown in the next section.}, which consists of two functions: $get$ (for getting the head of an input list) and $put$ (for reflecting the output to the head of the input). Figure \ref{fig:eval-phead} shows an example of the bidirectional behavior of $phead$.
Let $[1,2]$ be the original source $s$.
$get$ is a projection: $get$ of $phead$ picks the first element of the given original source $[1,2]$ and returns $1$ as a view $v$.
%\texttt{get pHead [1,2] = 1}.
%After the view \texttt{v} is obtained, the user can modify the view.
%In this case, the user
%modified the view
%from $1$ to $100$.
%From this updated view, how can the source be updated?
%This is ``the view update problem.''
%To solve this problem, BX provides $put$, an update function on the original source.
%In this example,
Supposing that the view is updated to $100$,
$put$ of $phead$ will construct a new source $s'$ of ${[100,2]}$ from the updated view ($v'$) of $2$ and the original source ($s$) of $[1,2]$.
%Note that the update to \texttt{s'} uses \texttt{s} and \texttt{v'} as input.
%\texttt{put pHead [1,2] 8 = [8,2]}

\begin{figure}[!t]
  \begin{minipage}{0.3\textwidth}
    \centering
    \includegraphics[height=3.5cm]{./fig/fig1.eps}
    \caption{Evaluating $phead$}
    \label{fig:eval-phead}
  \end{minipage}\hfill
  \begin{minipage}{0.7\textwidth}
    \centering
    \includegraphics[height=3.5cm]{./fig/fig2.eps}
    \caption{Evaluating $phead \circ phead \circ phead$}
    \label{fig:eval-comp-phead}
  \end{minipage}
\end{figure}



The composition of BX programs is a fundamental construct to build more complex BX programs \cite{Bohannon06relationallenses:, Bohannon:2008:BRL:1328438.1328487}. Let $bx_1$ (defined by $get_{bx_1}$ and $put_{bx_1}$) and $bx_2$ (defined by $get_{bx_2}$ and $put_{bx_2}$) be two bidirectional programs, then their composition $bx_1 \circ bx_2$ is defined by
\begin{align}
get_{bx_1 \circ bx_2}~s &= get_{bx_2} (get_{bx_1}~s)\\
put_{bx_1 \circ bx_2}~s~v' &= put_{bx_1} s~(put_{bx_2}~ (get_{bx_1}~s)~v')
\end{align}
One feature of this composition is that $put_{bx_1 \circ bx_2}$ needs to call $get_{bx_1}$ to compute the intermediate result for $put_{bx_2}$ to use, which would introduce an efficient problem if we compute $put$ for composition of many bidirectional programs. Generally, for a composition of $O(n)$ bidirectional programs, we need to call $get$ for $O(n^2)$ times. To be concrete, consider the evaluation of the following composition (which will be used as our running example in this paper):
\[
lp3 = (phead \circ phead) \circ phead
\]
which is illustrated by Figure \ref{fig:eval-comp-phead} with the original source $s$ being ${[[[1,2],[3]],[4]]}$ and the updated view ${100}$.
To obtain the final updated source $s'$, $put$ for $lp3$ needs to evaluate $put$ of $phead$ three times. The first is from $i_2$ and $v'$ to obtain $i'_2$, which needs to call $get$ twice to compute $i_2$; the second is from $i_1$ and $i'_2$ to obtain $i'_1$, which needs to call $get$ once, and the last is from $s$ and $i'_1$ to obtain $s'$, which is just a direct $put$ computation.

%In BX programs, composition of programs is essential for writing various interesting evaluations.
%Because recursions are achieved by compositions, large number of programs include compositions.
%In the BX lens papers \cite{Bohannon06relationallenses:, Bohannon:2008:BRL:1328438.1328487} compositions are also known as central part of BX programs.
%In this section we use a small BX program that contains compositions as a running example, because the programs including recursions can be complicated easily. The running examples are two versions of two composition of $phead$: ($phead \circ phead) \circ phead$ (we call lp3) and $phead \circ (phead \circ phead$) (we call rp3).


%For more various evaluations, we can use the composition of programs. Let us consider another example, a composition of 3 $pHead$s: $pHead \circ pHead \circ pHead$. Figure \ref{fig:eval-comp-phead} illustrates the evaluation of this program where the original source \texttt{s} is \texttt{[[[1,2],[3]],[4]]} and the updated view is \texttt{100}.
%By repetitious evaluation of $get$s of $pHead$, we can obtain the view $v$, \texttt{1}. To obtain the final updated source \texttt{s'}, we need to evaluate $put$ three times. The first put evaluation is from \texttt{i2} and \texttt{v'} and we can obtain \texttt{i2'}.


%By repetitious evaluation of $get$s of $phead$, we can obtain the view $v$, ${1}$. To obtain the final updated source \texttt{s'}, we need to evaluate $put$ three times. The first put evaluation is from \texttt{i$_2$} and \texttt{v'} and we can obtain \texttt{i$_2$'}.

%First, let us see the standard evaluation method of compositions: ``not keeping any intermediate states and obtaining them by evaluation when they are needed.'' In this example, ``intermediate states'' are \texttt{i$_1$} and \texttt{i$_2$}. This method is used in a BX language, BiGUL \cite{Ko:2016:BFV:2847538.2847544,Ko:2017:ABB:3177123.3158129}. The merit of this method is clear semantics and easy to implement. The disadvantage is the number of $get$s will be quadratic when the BX programs' compositions are left associative. In the evaluation of lp3, two $get$s are required for obtaining \texttt{i$_2$} and one $get$ is required for obtaining \texttt{i$_1$}. In total, three $get$s are evaluated. We explain this method in Section~\ref{sect:minbigul}.

One direct solution to avoid this repeated computation of $get$ is to compute the composition in a right associative manner. For instance, if we transform  $lp3$ to $rp3$:
\[
rp3 = phead \circ (phead \circ phead)
\]
then the $put$ for $rp3$ only needs to compute $get$ of $phead$ twice, one time less than that for $lp3$.
 However, this transformation is not always easy to do. For instance, let us consider $breverse$, a bidirectional version of the traditional 'reverse' program for reversing a list. It is defined using $\mbox{\it bfoldr}$, a bidrecrtional version of the traditional $\mbox{\it foldr}$, whose definition  is shown in the last part of Section~\ref{sect:minbigul}. Informally, $\mbox{\it bfoldr}$ is a recursive bidirectional program defined in a way like
 \[
 \mbox{\it bfoldr}~bx~\cdots = \cdots  (\mbox{\it bfoldr}~\cdots) \circ bx \cdots
 \]
where the composition is inherently left associative, and the number of composition is dynamically determined by the length of the source list. This make it hard to do the above transformation statically.


%A part of the definition of $\mbox{\it bfoldr}$, we have $(\product{Replace}{(\mbox{\it bfoldr} \ bx)}) \circ bx$. In this program, recursion occurs on the left-hand side of composition. Because of priority, it is impossible to transform this program to right associative compositions. At the same time, we can not apply program fusion \cite{Wadler:1988:DTP:80099.80104} to this program because actual program is dynamically produced by recursions depending on the length of the input list. Therefore, it is important to have a fast evaluation method for left associative compositions.

\begin{figure}[!t]
  \centering
  \includegraphics[height=5cm]{./fig/fig3.eps}
  \caption{Evaluating $phead \ \circ \ phead \ \circ \ phead$ by keeping intermediate states}
  \label{fig:eval-comp-phead-2}
\end{figure}


In this paper, we made the first attempt for seriously considering efficiency of evaluation of BX compositions, and
solve the problem by introducing two methods based on memoization to gain fast evaluation for (left associative) BX compositions.
The first method uses memoization simply: ``keeping all intermediate states and using them when they are needed''. This can avoid repeated computation of  $get$s, therefore it improves the runtime of many BX programs. However, this simple memoization needs to keep and search values in a table, which may introduce big cost for large inputs. We will explain this method in Section~\ref{sect:minbigulm}.

To treat large inputs, we propose the second method based on memoization: ``keeping complements in a closure and using them when they are needed''. Complements are smaller program fragments than the original intermediate states.
%Therefore this strategy solves the previous two problems.
%Readers might already know that it is a hard problem to obtain complements in general. In our work, thanks to the following finding, it is not hard:
%
%\vspace{2mm}
%In very-well behaved \cite{Foster:2007:CBT:1232420.1232424} BX,% programs,
%$put$ is a complement function for $get$.
%, $get$ can be a complement function for $put$.
%\vspace{2mm}
%
For obtaining complements, we tuple $put$ and $get$, and produce a new function $pg$. Because $put$ produces new complements for $get$, we can shrink the size.
As an example, let us consider the previous example again: $phead \circ phead \circ phead$ in Figure \ref{fig:eval-comp-phead-2}. $c_1$ and $c_2$ are complements. $d_1$ and $d_2$ are valid views for $s$ and $i_1$. Here, two points are worth noting. First, after evaluation of the first $pg$, we do not need to keep the original source $s$, because all its information is in $c_1$ and $i_1$. Second, complements are smaller than the intermediate states in the previous figure.
%This evaluation looks better than previous two strategies: this does not require repeated evaluation and require smaller storage than the original sources.
Actually, the simple combined $pg$ is not effective for left associative compositions, because this requires other two $put$ in the right part of the figure. To achieve efficient evaluation, we use two techniques, lazy update and lazy evaluation. We explain the second method and optimization in Section~\ref{sect:xpg}.

Both methods have been fully implemented for minBiGUL, a core bidirectional language. The experimental results show that our methods are much faster than the original evaluation strategy.
We give detailed experimental results in Section~\ref{sect:experiments}, discuss related work in Section~\ref{sect:related}, and conclude in Section~\ref{sect:conclusion}.


%The contributions of this paper are the following.

%\begin{itemize}
%\item This is the first attempt for seriously considering efficiency of evaluation of BX compositions. Our methods are better than or similar to the results by the original method in BiGUL for all test cases.
%\item We show that memoization is effective for achieving evaluation efficiency in BX languages.
%\item In BX, as far as authors know optimization by $get$ and $put$ more tight is the first attempt.
%\item Although we focus on a BX language BiGUL in this paper, these techniques can be potentially used in other BX languages.
  % \begin{itemize}
    % Although the first strategy is implemented in BiGUL, the second and third strategies are introduced by this paper.
%  \item Thanks to introduction of the strategies, it is possible to compare the evaluation strategies.
%  \item Improvement of evaluation efficiency
%  \end{itemize}
%\item Approaches used in strategy 3
%  \begin{itemize}
%  \item
% \item Optimization tequniques by tupling, and lazy update
%  \item These tequniques can be potentially used in other BX languages
%  \end{itemize}
%\end{itemize}


\section{Bidirectional Programming Language: minBiGUL}

minBiGUL, our target language in this paper, is a subset of BiGUL which is a simple yet powerful putback-based bidirectional language. BiGUL supports two transformations: a forward transformation $get$ producing a view from a source and a backward transformation $put$ taking a source and a modified view to produce an updated source. Intuitively, if we have a BiGUL program $bx$, these two transformations are following functions:\\
    $\tab \getbxinline{bx} : s \to v,
    \tab \putbxinline{bx} : s * v \to v$

BiGUL is well-behaved [?] since two functions $\putbxinline{bx}$ and $\getbxinline{bx}$ satisfy the round-trip laws as follows:\\
    $\tab \putbx{bx}{s}{(\getbx{bx}{s})} = s \qquad [\textsc{GetPut}]\\
    \tab \getbx{bx}{(\putbx{bx}{s}{v})} = v \qquad [\textsc{PutGet}]$

The \textsc{GetPut} [?] law means that if there is no change to the view, there should be no change to the source. The \textsc{PutGet} [?] law means that we can recover the modified view by applying the forward transformation to the updated source.

minBiGUL inherited from BiGUL also supports transformations $put$ and $get$ which are satisfy two above laws. In addition, we restrict adaptive cases of BiGUL on minBiGUL. Then $put$ and $get$ satisfy one more law, \textsc{PutPut} [?], like the following:\\
    $\tab \putbx{bx}{(\putbx{bx}{s}{v'})}{v} = \putbx{bx}{s}{v} \qquad [\textsc{PutPut}]$

The \textsc{PutPut} law means that a source update should overwrite the effect of previous source updates. Due to the satisfaction of three laws, \textsc{GetPut}, \textsc{PutGet} and \textsc{PutPut}, minBiGUL is very well-behaved [?].

\subsection{Syntax}

The syntax of minBiGUL is briefly written as follows:

$\tab bx := Skip \ h \
        | \ Replace \
        | \ Prod \ bx_1 \ bx_2 \
        | \ RearrS \ f_1 \ f_2 \ bx \
        | \ RearrV \ g_1 \ g_2 \ bx \\
    \qtab | \ Case \ cond_{sv} \ cond_{s} \ bx_1 \ bx_2 \
        | \ Compose \ bx_1 \ bx_2$

A minBiGUL program may be either a skip of a function or a replacement or a product of two minBiGUL programs or a source/view rearrangement or a case combinator without adaptive cases or a composition of some minBiGUL programs. We use numbers, pairs and lists to construct the program input including the source and/or the view.

For source/view rearrangement, BiGUL uses just one lambda expression to express how to deconstruct as well as reconstruct data. It is a kind of bijection. However, to be able to implement it in OCaml, the environment used for developing minBiGUL and solutions in the paper, we need give two functions which one is the inverse of the other. In the above syntax, $f_2 = f_1^{-1}$ and $g_2 = g_1^{-1}$.

To help make demonstration more direct, we provide the following alternatives representation: $Prod \ bx_1 \ bx_2 \equiv bx_1 \times bx_2, \ Compose \ bx_1 \ bx_2 \equiv bx_1 \circ bx_2$. In general, $\circ$ has a higher priority than $\times$. Their associativity precedence can be either left or right or mixture, but are not set by default. We need to explicitly write programs that use these operators.

\subsection{Semantics}

\begin{multicols}{2}
    \begin{definition}
        $\putbx{bx}{s}{v}$

        $\putbx{Skip \ h}{s}{v} = \\
            \tab \sif{h \ s = v}{s}{\text{fail}}$
    
        $\putbx{Replace}{s}{v} = v$
    
        $\putbx{\product{bx_1}{bx_2}}{(s_1, s_2)}{(v_1, v_2)} = \\
            \tab ((\putbx{bx_1}{s_1}{v_1}),(\putbx{bx_2}{s_2}{v_2}))$
    
        $\putbx{\rearrs{f_1}{f_2}{bx}}{s}{v} = \\
            \tab f_2 \ (\putbx{bx}{(f_1 \ s)}{v})$
    
        $\putbx{\rearrv{g_1}{g_2}{bx}}{s}{v} = \\
            \tab \putbx{bx}{s}{(g_1 \ v)}$
    
        $\putbx{\casebx{cond_{sv}}{cond_{s}}{bx_1}{bx_2}}{s}{v} = \\
            \tab \text{if} \ {cond_{sv} \ s \ v} \\
            \tab \text{then} \ s' \Leftarrow \putbx{bx_1}{s}{v} \\
            \tab \text{else} \ s' \Leftarrow \putbx{bx_2}{s}{v} \\
            \tab \text{fi} \ cond_{s} \ s'; \ \text{return} \ s'$
    
        $\putbx{bx_1 \circ bx_2}{s}{v} = \\
            \tab \putbx{bx_1}{s}{(\putbx{bx_2}{(\getbx{bx_1}{s})}{v})}$
    \end{definition}
\columnbreak
    \begin{definition}
        $\getbx{bx}{s}$

        $\getbx{Skip \ h}{s} = \\ 
            \tab h \ s$

        $\getbx{Replace}{s} = s$

        $\getbx{\product{bx_1}{bx_2}}{(s_1,s_2)} = \\
            \tab ((\getbx{bx_1}{s_1}),(\getbx{bx_2}{s_2}))$

        $\getbx{\rearrs{f_1}{f_2}{bx}}{s} = \\ 
            \tab \getbx{bx}{(f_1 \ s)}$

        $\getbx{\rearrv{g_1}{g_2}{bx}}{s} = \\ 
            \tab g_2 \ (\getbx{bx}{s})$

        $\getbx{\casebx{cond_{sv}}{cond_{s}}{bx_1}{bx_2}}{s} = \\
            \tab \text{if} \ {cond_{s} \ s} \\
            \tab \text{then} \ v' \Leftarrow \getbx{bx_1}{s} \\ 
            \tab \text{else} \ v' \Leftarrow \getbx{bx_2}{s} \\ 
            \tab \text{fi} \ {cond_{sv} \ s \ v'}; \ \text{return} \ v'$

        $\getbx{bx_1 \circ bx_2}{s} = \\ 
            \tab \getbx{bx_2}{(\getbx{bx_1}{s})}$
    \end{definition}
\end{multicols}

We define the semantics of $put$ and $get$ as in definitions 1 and 2 respectively. Instead of using $v'$ in the $put$ direction like figures \ref{fig:eval-phead}, \ref{fig:eval-comp-phead} and \ref{fig:eval-comp-phead-2}, we just simply use notation $v$ as the updated view. The later definitions also use this convention. \textcolor{red}{$Leftarrow$ means assignments.}

In the above definitions, we use if-then-else-fi statements to express semantics of $\putbxinline{Case}$ and $\getbxinline{Case}$. This statement is useful to describe many functions related to $Case$ in this paper. Statement (if $E_1$ then $X_1$ else $X_2$ fi $E_2$) means if the test $E_1$ is true, the statement $X_1$ is executed and the assertion $E_2$ must be true, otherwise, i.e. $E_2$ is false, the statement $X_2$ is executed and the assertion $E_2$ must be false. If the values of $E_1$ and $E_2$ are distinct, the if-then-else-fi structure is undefined. We can write the equivalent if-then-else statement as follows:

$\tab \text{if } E_1 \text{ then } X_1 \text{ else } X_2 \text{ fi } E_2; S\\
\tab \equiv \text{if }E_1 = true \text{ then } \{ X_1; \text{ if } E_2 = true \text{ then } S \text{ else assert } false; \}\\
    \qtab \text{else } \{ X_2; \text{ if } E_2 = false \text{ then } S \text{ else assert } false; \}$

% Next, let's take a look at some examples to better understand about minBiGUL. We start with quite obvious things as follows:

% $\tab \putbx{Skip \ (\lambda x.(x*x))}{10}{100} = 10 \qtab \ \getbx{Skip \ (\lambda x.(x*x))}{10} = 100\\
%     \tab \putbx{Skip \ (\lambda \_.())}{1}{()} = 1 \qtabs{3} \getbx{Skip \ (\lambda \_.())}{1} = ()\\
%     \tab \putbx{Replace}{1}{100} = 100 \qtabs{3} \getbx{Replace}{1} = 1
% $

We now consider the definition of $phead$ in minBiGUL:

$\tab phead = RearrS \ f_1 \ f_2 \ bx_s \ \text{where: } f_1 = \lambda (s::ss).(s,ss), \, f_2 = \lambda (s,ss).(s::ss),\\
    \qtab bx_s = RearrV \ g_1 \ g_2 \ bx_v \ \text{where: } g_1 = \lambda v.(v,()), \, g_2 = \lambda (v,()).v,\\
        \qtabs{2} bx_v = \product{Replace}{(Skip \ (\lambda \_.())}$

The above program rearranges the source, a non-empty list, to a pair of its head element $s$ and its tail $ss$, and the view to a pair $(v, ())$, then we can use $v$ to replace $s$ and $()$ to keep $ss$. Intuitively, $\putbx{phead}{s_0}{v_0}$ returns a list whose head is $v_0$ and tail is the tail of $s_0$, and $\getbx{phead}{s_0}$ returns the head of the list $s_0$. For instance, $\putbx{phead}{[1,2,3]}{100} = [100,2,3]$ and $\getbx{phead}{[1,2,3]} = 1$. If we wanna update the head element of the head element of a list of lists by using the view, we can define a composition like $phead \circ phead$. For example:\\
\tab $\putbx{phead \circ phead}{[[1,2,3],[\,],[4,5]]}{100} = [[100,2,3],[\,],[4,5]]$\\
\tab $\getbx{phead \circ phead}{[[1,2,3],[\,],[4,5]]} = 1$


\section{Adding Memoization: minBiGUL$_m$} \label{sect:minbigulm}

When evaluating the composition of several BX programs, 
% (and the same holds for 
% full BiGUL), 
% all BX languages), 
the same $get$s are evaluated repeatedly.
% because no intermediate states are kept during the evaluation. 
This problem was illustrated in Figure~\ref{fig:eval-comp-phead}. To avoid reevaluating $get$s, and as our first approach to avoid this inefficiency, we introduce memoization in the minBiGUL interpreter.
% can save 
To keep it simple, the intermediate state of a composition is saved in a key-value table where the key is a pair of program $bx$ and source $s$, and the value is the result of evaluating $\getbx{bx}{s}$.
% the first time. 
% Then, instead of recomputing the value, the value in the table is used.
Later the value in the table is used instead of recomputing it. 

The memoizing version, minBiGUL$_m$, needs only two modifications: $get_m$ and $put_m$ (Definitions~\ref{def:putm} and~\ref{def:getm}). The auxiliary function, $get_{mh}$, is used to retrieve the value of an already evaluated $get$ from the table or to compute an unevaluated $get$ by $get_m$ and store the key-value tuple in the table (Definition~\ref{def:getmh}).

\begin{definition} \label{def:putm} $\text{Memo version of } put$

    \noindent $\putm{bx}{s}{v} = \match{bx}\\
        \tab \vert \ bx_1 \circ bx_2 \to \putm{bx_1}{s}{(\putm{bx_2}{(\getmh{bx_1}{s})}{v})} \\
        \tab \vert \ \_ \to \text{similar to } put$
\end{definition}

\begin{multicols}{2}
    \begin{definition} \label{def:getm} $\text{Memo version of } get$

        \noindent $\getm{bx}{s} = \match{bx}\\
            \tab \ \vert \ bx_1 \circ bx_2 \to\\
            \tabs{2} \textnormal{try } (\text{Hashtbl.find} \ table_g \ (bx, s))\\
            \tabs{2} \textnormal{with } \text{Not\_found } \to\\
                \tabs{3} v \Leftarrow \getmh{bx_2}{(\getmh{bx_1}{s})}\\
                \tabs{3} \text{Hashtbl.add } \ table_g \ (bx, s) \ v\\
                \tabs{3} v \\
            \tab \ \vert \ \_ \to \text{similar to } get$
    \end{definition}
\columnbreak
    \begin{definition} \label{def:getmh} $\text{Interface for } get_m$

        \noindent $\getmh{bx}{s} =\\
            \tab \textnormal{try } (\text{Hashtbl.find} \ table_g \ (bx, s))\\
            \tab \textnormal{with } \text{Not\_found } \to\\
                \tabs{2} v \Leftarrow \getm{bx}{s}\\
                \tabs{2} \match{bx} \\
                \tabs{2} \ \vert \ bx_1 \circ bx_2 \to v \\
                \tabs{2} \ \vert \ \_ \to \text{Hashtbl.add } \ table_g \ (bx, s) \ v\\
                \qtabs{2} \ v$
    \end{definition}
\end{multicols}

% Note as a special point 
% It should be noted 
Note in particular 
that the interpreter does not save all states when evaluating a program, only the intermediate states of a composition. When evaluating $\putm{bx_1 \circ bx_2}{s}{v}$, the function $get_{mh}$ is used, not $get$ or $get_m$.
% appears to indicate the saving. 
When evaluating $\getm{bx_1 \circ bx_2}{s}$, the key ($bx,s$), where $bx \equiv bx_1 \circ bx_2$, is looked up in the table and the corresponding value is used for the next steps in the evaluation. If there is no such key, the value $v$ will be calculated by using $get_{mh}$. This key-value insertion is necessary because the interpreter may later need to reevaluate $\getm{bx_1 \circ bx_2}{s}$.
% reevaluate <--> evaluate one more time.


\section{Tupling and Lazy Updates: xpg} \label{sect:xpg}
\subsection{Tupling: pg} \label{sect:pg}

Another solution for saving intermediate states is tupling. If $put$ and $get$ are evaluated simultaneously, it is potential to reduce the number of recomputed $get$s. The following function, $pg$, accepts the pair of a source and a view as the input to produce a new pair that contains the actual result of the corresponding minBiGUL program.

\begin{definition}
    $\pg{bx}{s}{v} = (\putbx{bx}{s}{v}, \getbx{bx}{s})$
\end{definition}

% $pg$ is an involution that is self-inverse. An involution is a function $f$ that satisfies $f(f(x)) = x \text{ for all } x \text{ in the domain of } f$.

% \begin{proof}
% $pg [\![bx]\!] \ (\pg{bx}{s}{v}) \\
%     \tab = \pg{bx}{(\putbx{bx}{s}{v})}{(\getbx{bx}{s})} \quad [pg \text{ definition}] \\
%     \tab = (put \ [\![bx]\!] \ (\putbx{bx}{s}{v}) \ (\getbx{bx}{s}), \getbx{bx}{(\putbx{bx}{s}{v})})  \quad [pg \text{ definition}] \\
%     \tab = (put \ [\![bx]\!] \ (\putbx{bx}{s}{v}, \getbx{bx}{s}), v) \quad [\textsc{PutGet}] \\
%     \tab = (put \ [\![bx]\!] \ (s, \getbx{bx}{s}), v) \quad [\textsc{PutPut}] \\
%     \tab = (s, v) \quad [\textsc{GetPut}]$
% \end{proof}

\noindent Now, let us see how we construct $pg$ recursively.

    \noindent $\tabs{2} \pg{Skip \ h}{s}{v} \overset{1}{=} (\sif{h \ s = v}{s}{\textit{undefined}}, h \ s) \\
        \tab \qtab \overset{2}{=} \sif{h \ s = v}{(s, h \ s)}{\textit{undefined}} \\
        \tab \qtab \overset{3}{=} \sif{h \ s = v}{(s, v)}{\textit{undefined}}$
        
The first equality is simply based on the definitions of $pg$, $\putbxinline{Skip \ h}$ and $\getbxinline{Skip \ h}$. The second one tuples two results of $put$ and $get$ in the body of the if-expression. This is a trick since in some cases, the result of $pg$ may be undefined although there is no undefined when evaluating $\getbxinline{Skip \ h}$. The last equality is relatively obvious.

    \noindent $\tabs{2} \pg{Replace}{s}{v} = (v, s)$\\
    $\tabs{2} \pg{\product{bx_1}{bx_2}}{(s_1,s_2)}{(v_1,v_2)}\\
        \tab \qtab \overset{1}{=} ((\putbx{bx_1}{s_1}{v_1}),(\putbx{bx_2}{s_2}{v_2}), (\getbx{bx_1}{s_1}), (\getbx{bx_2}{s_2})) \\
        \tab \qtab \overset{2}{=} (s_1, v_1) \Leftarrow \pg{bx_1}{s_1}{v_1}; \\
            \qtab \tabs{2} \ \, (s_2, v_2) \Leftarrow \pg{bx_2}{s_2}{v_2}; \\
            \tab \qtabs{2} ((s_1,s_2), (v_1,v_2))$\\
    $\tabs{2} \pg{\rearrs{f_1}{f_2}{bx}}{s}{v} \overset{1}{=} (f_2 \ (\putbx{bx}{(f_1 \ s)}{v}), \getbx{bx}{(f_1 \ s)}) \\
        \tab \qtab \overset{2}{=} (s, v) \Leftarrow \pg{bx}{f_1 \ s}{v};\\
            \tab \qtabs{2} (f_2 \ s, v)$\\
    $\tabs{2} \pg{\rearrv{g_1}{g_2}{bx}}{s}{v} \overset{1}{=} (\putbx{bx}{s}{(g_1 \ v)}, g_2 \ (\getbx{bx}{s})) \\
        \tab \qtab \overset{2}{=} (s, v) \Leftarrow \pg{bx}{s}{g_1 \ v}; \\
        \tab \qtabs{2} (s, g_2 \ v)$

Constructions of $pg$ for the replacement, the product and the source/view rearrangements are very clear when just paring $put$ and $get$ respectively, then doing basic changes.\\

    \noindent $\tabs{2} \pg{\casebx{cond_{sv}}{cond_{s}}{bx_1}{bx_2}}{s}{v} \\
    \tabs{3} \overset{1}{=} (\text{if} \ {cond_{sv} \ s \ v} \qtabs{3} \tab \text{if} \ {cond_{s} \ s} \\
    \tabs{2} \qtab \text{then} \ s' \Leftarrow \putbx{bx_1}{s}{v} \qtab \tab \, \text{then} \ v' \Leftarrow \getbx{bx_1}{s} \\
    \tabs{2} \qtab \text{else} \ s' \Leftarrow \putbx{bx_2}{s}{v} \qtab \tab \ \, \text{else} \ v' \Leftarrow \getbx{bx_2}{s}\\
    \tabs{2} \qtab \text{fi} \ cond_{s} \ s'; \text{ return } s' \qtab , \tab \ \ \text{fi} \ {cond_{sv} \ s \ v'}; \text{ return } v')\\
    \tabs{3} \overset{2}{=} \text{if} \ {cond_{sv} \ s \ v} \ \&\& \ {cond_{s} \ s}\\
    \tabs{2} \qtab \text{then} \ (s', v') \Leftarrow \pg{bx_1}{s}{v}\\
    \tabs{2} \qtab \text{else} \ (s', v') \Leftarrow \pg{bx_2}{s}{v}\\
    \tabs{2} \qtab \text{fi} \ cond_{s} \ s' \ \&\& \ cond_{sv} \ s \ v'; \text{ return } (s',v')$

A restriction for $\pginline{Case}$ needs to be introduced here. We know that there is one entering condition and one exit condition when evaluating $\putbxinline{Case}$ as well as $\getbxinline{Case}$. If a tupling occurs, there will be 4 combinations from these conditions. This means two entering conditions of $\putbxinline{Case}$ and $\getbxinline{Case}$ are not always simultaneously satisfied. The evaluated branches are distinct in the $put$ and $get$ directions for combinations $((cond_{sv} \ s \ v) \ \&\& \ (not (cond_{s} \ s)))$ and $((not (cond_{sv} \ s \ v)) \ \&\& \ (cond_{s} \ s))$, which are restricted in this paper. This does not happen for the others which is used in the construction of $\pginline{Case}$.

    \noindent $\tabs{2} \pg{bx_1 \circ bx_2}{s}{v} \\
    \tabs{3} \overset{1}{=} (\putbx{bx_1}{s}{(\putbx{bx_2}{(\getbx{bx_1}{s})}{v})}, \getbx{bx_2}{(\getbx{bx_1}{s})}) \\
    \tabs{3} \overset{2}{=} v_1 \Leftarrow \getbx{bx_1}{s}; \qtabs{3} \ \, \overset{3}{=} (s_1, v_1) \Leftarrow \pg{bx_1}{s}{dummy};\\
        \tabs{2} \qtab (s_2, v_2) \Leftarrow \pg{bx_2}{v_1}{v}; \qtabs{2} (s_2, v_2) \Leftarrow \pg{bx_2}{v_1}{v};\\
        \tabs{2} \qtab (s_3, v_3) \Leftarrow \pg{bx_1}{s}{s_2}; \qtabs{2} (s_3, v_3) \Leftarrow \pg{bx_1}{s}{s_2};\\
            \qtab \tabs{3} (s_3, v_2) \qtabs{6} (s_3, v_2)\\
    \tabs{3} \overset{4}{=} (s_1, v_1) \Leftarrow \pg{bx_1}{s}{construct\_dummy \ s}; \\
        \tabs{2} \qtab (s_2, v_2) \Leftarrow \pg{bx_2}{v_1}{v}; \\
        \tabs{2} \qtab (s_3, v_3) \Leftarrow \pg{bx_1}{s_1}{s_2}; \\
            \qtab \tabs{3} (s_3, v_2)$
            
The construction of $\pginline{bx_1 \circ bx_2}$ can be considered as the soul of the $pg$ function. The first two equalities comes from mentioned definitions and some basic transformations. The third one rewrites $v_1 \Leftarrow \getbx{bx_1}{s}$ into $(s_1, v_1) \Leftarrow \pg{bx_1}{s}{dummy}$. This is possible when we consider $\getbx{bx_1}{s}$ as the second element of $\pg{bx_1}{s}{dummy}$ where $dummy$ is a special value that makes the $put \, [\![bx_1]\!]$ valid.
%We can construct a $dummy$ from the source by \textsc{GetPut} law in some cases or require programmers to give a valid view.
Since there is no real view, this $dummy$ is necessary. The last equality changes two. First is a change from $dummy$ to an application $construct\_dummy \ s$. This means we need a way to construct $dummy$ from the current source, $s$. We can require programmers to give this $construct\_dummy$ function. Second is a substitution under the \textsc{PutPut} law on $s$ of $(s_3, v_3) \Leftarrow \pg{bx_1}{s}{s_2}$ by $s_1$ which equals to $\putbx{bx_1}{s}{(construct\_dummy \ s)}$. $v_3$ in the result pair $(s_3,v_3)$ equals $dummy$. So we realize that there is no loss information when computing a composition.

\subsection{Lazy Update: cpg} 

When evaluating $\pginline{bx_1 \circ bx_2}$, there are three $pg$ calls, of which twice for $\pginline{bx_1}$ and once for $\pginline{bx_2}$. If a given program is a left associative composition, the number of $pg$ calls will be exponential. Therefore, the runtime inefficiency is inevitable. To solve that, we introduce a new function, $cpg$, accumulates updates on the source and the view. $\cpg{bx}{ks}{kv}{s}{v}$ is an extension of $\pg{bx}{s}{v}$ where $ks$ and $kv$ are continuations used to hold the modification information, and $s$ and $v$ are used to keep evaluated values same as $pg$. The output of this function is a 4-tuple $(ks, kv, s, v)$. To be more convenient for presenting the definition of $cpg$ as well as the other functions later, we provide some following utility functions:\\ $fst = \lambda (x_1,x_2). x_1 , \ snd = \lambda (x_1,x_2). x_2 , \ con = \lambda ks_1. \lambda ks_2. \lambda x. ((ks_1 \ x),(ks_2 \ x))$

\begin{definition}
$\cpg{bx}{ks}{kv}{s}{v}$

    \noindent $\cpg{Skip \ h}{ks}{kv}{s}{v} = \textnormal{ if } h \ s = v \text{ then } (ks, kv, s, v) \textnormal{ else } \textit{undefined}$

    \noindent $\cpg{Replace}{ks}{kv}{s}{v} = (kv, ks, v, s)$

    \noindent $\cpg{\product{bx_1}{bx_2}}{ks}{kv}{s}{v} =\\
        \tab (ks_1, kv_1, s_1, v_1) \Leftarrow \cpg{bx_1}{fst \circ ks}{fst \circ kv}{fst \ s}{fst \ v};\\
        \tab (ks_2, kv_2, s_2, v_2) \Leftarrow \cpg{bx_2}{snd \circ ks}{snd \circ kv}{snd \ s}{snd \ v};\\
        \qtab (con \ ks_1 \ ks_2, con \ kv_1 \ kv_2, (s_1,s_2), (v_1,v_2))$

    \noindent $\cpg{RearrS \ f_1 \ f_2 \ bx}{ks}{kv}{s}{v} =\\
        \tab (ks, kv, s, v) \Leftarrow \cpg{bx}{f_1 \circ ks}{kv}{f_1 \ s}{v};\\
        \qtab (f_2 \circ ks, kv, s, v)$

    \noindent $\cpg{RearrV \ g_1 \ g_2 \ bx}{ks}{kv}{s}{v} =\\
        \tab (ks, kv, s, v) \Leftarrow \cpg{bx}{ks}{g_1 \circ kv}{s}{g_1 \ v};\\
        \qtab (ks, g_2 \circ kv, ks, g_2 \ v)$

    \noindent $\cpg{Case \ cond_{sv} \ cond_{s} \ bx_1 \ bx_2}{ks}{kv}{s}{v} =\\
        \tab \textnormal{if} \ cond_{sv} \ s \ v \ \&\& \ cond_{s} \ s\\
        \tab \textnormal{then} \ (ks, kv, s', v') \Leftarrow \cpg{bx_1}{ks}{kv}{s}{v}\\
        \tab \textnormal{else} \ (ks, kv, s', v') \Leftarrow \cpg{bx_2}{ks}{kv}{s}{v}\\
        \tab \textnormal{fi} \ cond_{s} \ s' \ \&\& \ cond_{sv} \ s \ v'; \ \textnormal{return} \ (ks, kv, s', v')$

    \noindent $\cpg{bx_1 \circ bx_2}{ks}{kv}{s}{v} =\\
        \tab (ks_1, kv_1, \underline{s_1}, v_1) \Leftarrow \cpg{bx_1}{ks}{id}{s}{construct\_dummy \ s};\\
        \tab (ks_2, kv_2, s_2, v_2) \Leftarrow \cpg{bx_2}{kv_1}{kv}{v_1}{v};\\
            \qtab (ks_1 \circ ks_2, kv_2,  ks_1 \ s_2, v_2)$
\end{definition}

In the places where third and/or fourth argument ($s$ and $v$) are updated by applications, the computations are also accumulated in $ks$ and/or $kv$.
%Apart from the last construction, the others are quite similar to the corresponding ones of $pg$, but have some updates over the source and the view on accumulative functions $ks$ and $kv$ respectively.
Thanks to these accumulations, there are only two $cpg$ calls in $\cpginline{bx_1 \circ bx_2}$. The first call $\cpginline{bx_1}$ requires parameter $(ks, id, s, construct\_dummy \ s)$ where $s$ and $ks$ are corresponding to the source and the update over source. Since there is no real view here, we need to construct a dummy from the source same as $pg$. Then the continuation updating on this dummy should be initiated as an identity function. The first $cpg$ call is assigned to a 4-tuple $(ks_1, kv_1, s_1, v_1)$. In the next assignment, a 4-tuple $(ks_2, kv_2, s_2, v_2)$ is assigned by the second $cpg$ call which uses the input as $(kv_1, kv, v_1, v)$ where $kv_1$ and $v_1$ are obtained from the result of the first assignment, and $kv$ and $v$ come from the input. It is relatively similar to the second $pg$ call assignment in $\pginline{bx_1 \circ bx_2}$. After two $cpg$ calls, a function application, $ks_1 \ s_2$, is used to produce the updated source instead of calling recursively one more time like in $\pginline{bx_1 \circ bx_2}$.

Suppose that we have a source $s_0$ and a view $v_0$. The pair of the updated source and view $(s, v)$ where $s = \putbx{bx}{s_0}{v_0}$ and $v = \getbx{bx}{s_0}$ can be obtained using $cpg$ as follows:

\smallvspace
    $\tab s \Leftarrow s_0; v \Leftarrow v_0; (ks, kv, s, v) \Leftarrow \cpg{bx}{\lambda \_.s}{id}{s}{v};\\
        \tab \qtab (s; v)$
\smallvspace
        
In general, the beginning of a continuation should be an identity function. However, to be able to use the function application to get the result of $\cpginline{bx_1 \circ bx_2}$, the accumulative function on the source $s$ needs to be initiated as $\lambda \_.s$. This constant function helps to retain the discarded things in the source.



Note that, in $\cpginline{bx_1 \circ bx_2}$, $\underline{s_1}$ is redundant because this evaluated variable is not used in the later steps.
In the next session, we will optimize this redundancy.

% Suppose the beginning of continuations $ks$ and $kv$ are $ks_0$ and $id$ respectively. Let's consider $\cpg{phead \circ phead}{ks_0}{id}{s}{v}$ where $s = [[1,2,3], [\,], [4,5]]$ and $v = 100$. After the first two assignments in the definition of $cpg$ for the composition, we have: $ks_1 = f_2 \circ (con \ (fst \circ g_1 \circ id) \ (snd \circ f_1 \circ ks_0))$ and $s_2 = [100,2,3]$ where $f_1 = \lambda (s::ss).(s,ss)$, $f_2 = \lambda (s,ss).(s::ss)$, $g_1 = \lambda v.(v,())$. Then:

% $ks_1 \ s_2 = (f_2 \circ (con \ (fst \circ g_1 \circ id) \ (snd \circ f_1 \circ ks_0))) \ s_2 \\
% \tab \quad = f_2 \ (\ (fst(g_1(id(s_2))) \ , \ snd(f_1(ks_0(s_2))))\ )\\
% \tab \quad = fst(g_1(id(s_2))) :: snd(f_1(ks_0(s_2))) = [100,2,3] :: snd(f_1(ks_0([100,2,3])))$

% If $ks_0$ is an identity function, $ks_1 \ s_2 = [100,2,3] :: [2,3]$. This is an unexpected result when we target it to be the updated source. If $ks_0 = \lambda \_.s$ where $s = [[1,2,3], [\,], [4,5]]$, the result will be what we want to see. $ks_1 \ s_2 = [100,2,3] :: [[\,], [4,5]] = [[100,2,3], [\,], [4,5]]$. Through the above example, using the continuation $ks$ as a constant function at the beginning contributes to retain the discarded things in the source.


\subsection{Lazy Computation: kpg}

The problem for $cpg$ lies in redundant computations during evaluating. For keeping away such computations, we introduce an extension named $kpg$. While $cpg$ evaluates values eagerly, $kpg$ does the opposite. Every values are evaluated lazily in a computation of $kpg$. The input of $\kpginline{bx}$ is expanded to a 6-tuple $(ks, kv, ks', kv', s, v)$ where $ks$ and $kv$ keep the modification information same as $cpg$, $s$ and $v$ hold evaluated values, and $ks'$ and $kv'$ are used for lazy evaluation of actual values. The output of this function is also a 6-tuple $(ks, kv, ks', kv', s, v)$.\\
Suppose that we have a source $s_0$ and a view $v_0$. The pair of the updated source and view $(s, v)$ where $s = \putbx{bx}{s_0}{v_0}$ and $v = \getbx{bx}{s_0}$ can be obtained using $kpg$ as follows:

\smallvspace
    $\tab s \Leftarrow s_0; v \Leftarrow v_0; (ks, kv, ks', kv', s, v) \Leftarrow \kpg{bx}{\lambda \_.s}{id}{id}{id}{s}{v};\\
        \tab \qtab (ks' \ s; kv' \ v)$
\smallvspace
        
The beginning of accumulative functions $ks'$ and $kv'$ are set as identity function, while $ks$ and $kv$ are initiated as the same with the corresponding ones in $cpg$.

\begin{definition}
$\kpg{bx}{ks}{kv}{ks'}{kv'}{s}{v}$

    \noindent $\kpg{Skip \ h}{ks}{kv}{ks'}{kv'}{s}{v} =\\
        \tab s \Leftarrow ks' \ s; \quad v \Leftarrow kv' \ v; \quad ks' \Leftarrow id; \quad kv' \Leftarrow id;\\
        \tab \textnormal{if} \ h \ s = v \ \textnormal{then} \ (ks, kv, ks', kv', s, v) \ \textnormal{else } \text{undefined}$

    \noindent $\kpg{Replace}{ks}{kv}{ks'}{kv'}{s}{v} = (kv, ks, kv', ks', v, s)$

    \noindent $\kpg{\product{bx_1}{bx_2}}{ks}{kv}{ks'}{kv'}{s}{v} =\\
        \tab s \Leftarrow ks' \ s; \quad v \Leftarrow kv' \ v; \quad ks' \Leftarrow id; \quad kv' \Leftarrow id;\\
        \tab (ks_1, kv_1, ks_1', kv_1', s_1, v_1) \Leftarrow \kpg{bx_1}{fst \circ ks}{fst \circ kv}{fst \circ ks'}{fst \circ kv'}{s}{v};\\
        \tab (ks_2, kv_2, ks_2', kv_2', s_2, v_2) \Leftarrow \kpg{bx_2}{snd \circ ks}{snd \circ kv}{snd \circ ks'}{snd \circ kv'}{s}{v};\\
        \qtab ( con \ ks_1 \ ks_2, con \ kv_1 \ kv_2, con \ (ks_1' \circ fst) \ (ks_2' \circ snd),\\
        \qtab con \ (kv_1' \circ fst) \ (kv_2' \circ snd), \\
        \qtab (s_1, s_2), (v_1,v_2))$

    \noindent $\kpg{RearrS \ f_1 \ f_2 \ bx}{ks}{kv}{ks'}{kv'}{s}{v} =\\
        \tab (ks, kv, ks', kv', s, v) \Leftarrow \kpg{bx}{f_1 \circ ks}{kv}{f_1 \circ ks'}{kv'}{s}{v};\\
        \qtab (f_2 \circ ks, kv, f_2 \circ ks', kv', s, v)$

    \noindent $\kpg{RearrV \ g_1 \ g_2 \ bx}{ks}{kv}{ks'}{kv'}{s}{v} =\\
        \tab (ks, kv, ks', kv', s, v) \Leftarrow \kpg{bx}{ks}{g_1 \circ kv}{ks'}{g_1 \circ kv'}{s}{v};\\
        \qtab (ks, g_2 \circ kv, ks', g_2 \circ kv', s, v)$

    \noindent $\kpg{Case \ cond_{sv} \ cond_{s} \ bx_1 \ bx_2}{ks}{kv}{ks'}{kv'}{s}{v} =\\
        \tab s \Leftarrow ks' \ s; \quad v \Leftarrow kv' \ v; \quad ks' \Leftarrow id ; \quad kv' \Leftarrow id;\\
        \tab \textnormal{if} \ cond_{sv} \ s \ v \ \&\& \ cond_{s} \ s\\
        \tab \textnormal{then} \ (ks, kv, ks', kv', s', v') \Leftarrow \kpg{bx_1}{ks}{kv}{ks'}{kv'}{s}{v}\\
        \tab \textnormal{else} \ (ks, kv, ks', kv', s', v') \Leftarrow \kpg{bx_2}{ks}{kv}{ks'}{kv'}{s}{v}\\
        \tab \underline{\textnormal{fi}} \ cond_s \ (ks' \ s') \ \&\& \ cond_{sv} \ s \ (kv' \ v'); \ \textnormal{return} \ (ks, kv, ks', kv', s', v')$

    \noindent $\kpg{bx_1 \circ bx_2}{ks}{kv}{ks'}{kv'}{s}{v} =\\
        \tab (ks_1, kv_1, \underline{ks_1'}, kv_1', \underline{s_1}, v_1) \Leftarrow \kpg{bx_1}{ks}{id}{ks'}{id}{s}{construct\_dummy \ s};\\
        \tab (ks_2, kv_2, ks_2', kv_2', s_2, v_2) \Leftarrow \kpg{bx_2}{kv_1}{kv}{kv_1'}{kv'}{v_1}{v};\\
        \qtab (ks_1 \circ ks_2, kv_2, ks_1 \circ ks_2', kv_2', s_2, v_2)$
\end{definition}

In $kpg$, basically, functions for the updates will be kept in $ks'$ and $kv'$.
$\kpginline{Skip}$ and $\kpginline{Case}$ are special cases, because they require the actual values for evaluation. Therefore we evaluate the values by application of $ks' \ s$ and $kv' \ v$ and update $ks'$ and $kv'$ with the identity functions.
%Except $\kpginline{Skip}$ and $\kpginline{Case}$ where $s$ and $v$ hold actual values, functions for the computation will be kept in $ks'$ and $kv'$.
When evaluating $\kpginline{bx_1 \circ bx_2}$, $\underline{ks_1'}$ and $\underline{s_1}$ in the result of the first assignment are still redundant but not evaluated.

Additionally we did two optimizations in $kpg$. First is in $\kpginline{bx_1 \times bx_2}$. The evaluation of $ks'$ and $kv'$ will be done independently in two assignments using $\kpginline{bx_1}$ and $\kpginline{bx_2}$. There may be same computations in $fst \circ ks'$ and $snd \circ ks'$ as well as $fst \circ kv'$ and $snd \circ kv'$.
To remove duplicate evaluations, we evaluate actual values in $s$ and $v$ before calling $\kpginline{bx_1}$.
%It is possible to evaluate actual values in $s$ and $v$ before calling $\kpginline{bx_1}$ to remove the redundancy like that. One more thing,
Second is in $\kpginline{Case}$. We need to evaluate $ks' \ s'$ and $kv' \ v'$ to check the \underline{fi} condition before returning the 6-tuple. Such evaluations can be done lazily to make programs run faster. We use the above small optimizations in our implementation.

\subsection{Combination of pg and kpg: xpg}
The most important thing when using $kpg$ to evaluate a composition $bx_1 \circ bx_2$ is keeping the dropped parts from the source in a function that is used to produce new source instead of using more recursive call. This is reflected in the first call $\kpginline{bx_1}$ and the last application in $\kpginline{bx_1 \circ bx_2}$. The second call $\kpginline{bx_2}$ just play a syntactic role while it requires many parameters. We can completely reduce this number to 2, including one for the source and one for the view as in a $pg$ call. This idea is realized by function $xpg$ as follows:

\begin{definition}
$\xpg{bx}{s}{v}$

    \noindent $\xpg{bx}{s}{v} = \textnormal{match } bx \textnormal{ with }\\
    \tab \vert \ bx_1 \circ bx_2 \to \\
        \tabs{2} (ks_1,kv_1, ks_1', kv_1', s_1, v_1) \Leftarrow \kpg{bx_1}{\lambda \_.s}{id}{id}{id}{s}{construct\_dummy \ s};\\
        \tabs{2} (s_2, v_2) \Leftarrow \xpg{bx_2}{kv_1' \ v_1}{v};\\
        \tabs{3} (ks_1 \ s_2, v_2)\\
    \tab \vert \ \_ \to \text{similar to } pg$
\end{definition}

Similar to $pg$, $\xpginline{bx}$ accepts a pair of the source and the view $(s,v)$ to produce the new pair. The constructions of $\xpginline{bx}$ when $bx$ is not a composition are the same as the ones of $\pginline{bx}$. Note that, $xpg$ is called recursively instead of $pg$. For $\xpginline{bx_1 \circ bx_2}$, we use two function calls and a function application to calculate the result. The first call and the function application come from $kpg$, while the second call is based on $pg$.


\section{Experiments}

We have implemented all methods described above in the same environment as follows: macOS 10.14.6, processor Intel Core i7 (2.6 GHz), RAM 16 GB 2400 MHz DDR4, OCaml 4.07.1. The OCaml runtime system options and garbage collection parameters are set as default. There is no type system in our implementation\footnote{\url{https://mega.nz/\#!yUsgyCxA!yrJfINiwtCwhmd4McYMz2xsFa-ljkgGKmMi839ut5gA}}, so instead of constructing some dummies which are necessary for $pg$, $cpg$, $kpg$ and $xpg$, we replace them by the updated view that helps a program in the $put$ direction valid.

\subsection{Test cases}

We have conducted 7 test cases shown in Table \ref{tab:test-cases}. Although the number of programs is not so much, they are not trivial and can be completely used in practice. In $s_r$ and $v_r$ are respectively updated source and view which are produced by applying $put$ and $get$ on original source $s_0$ and view $v_0$. This means: $s_r = \putbx{bx}{s_0}{v_0}$ and $v_r = \getbx{bx}{s_0}$, where $bx$ is present in the table. These $bx$ are left associative composition since we we are only interested in this kind of program in the paper. Please note that the results $s_r$ and $v_r$ do not depend on the associative style of the composition.

\begin{table}[hbt!]
    \centering
    \caption{Test cases}
    \label{tab:test-cases}
    \begin{tabular*}{\textwidth}{|l @{\extracolsep{\fill}}|l|l|c|c|c|c|}
        \hline
        \multirow{2}{*}{No} & \multicolumn{1}{c|}{\multirow{2}{*}{Name}} & \multicolumn{1}{c|}{\multirow{2}{*}{Type}} & \multicolumn{2}{c|}{Input} & \multicolumn{2}{c|}{Output} \\ \cline{4-7} 
        & \multicolumn{1}{c|}{} & \multicolumn{1}{c|}{} & \multicolumn{1}{c|}{$s_0$} & \multicolumn{1}{c|}{$v_0$} & \multicolumn{1}{c|}{$s_r$} & \multicolumn{1}{c|}{$v_r$} \\ \hline
        1 & lcomp-phead-ldata & straight line & $\underbrace{[[\ldots[1]\ldots]]}_{\text{n+1 times}}$ & 100 & $\underbrace{[[\ldots[100]\ldots]]}_{\text{n+1 times}}$ & 1 \\ \hline
        2 & lcomp-ptail & straight line & [1,$\ldots$,n+1] & [1,$\ldots$,10] & $s_0 \ @ \ v_0$ & [\ ] \\ \hline
        3 & lcomp-ptail-ldata & straight line & $\underbrace{[L,\ldots,L]}_{\text{n+1 times}}$ & $\underbrace{[L,\ldots,L]}_{\text{10 times}}$ & $\underbrace{[L,\ldots,L]}_{\text{n+11 times}}$ & [\ ] \\ \hline
        4 & lcomp-bsnoc & straight line & [1,$\ldots$,n-1] & [1,$\ldots$,n-1] & [1,$\ldots$,n-1] & [1,$\ldots$,n-1] \\ \hline
        5 & lcomp-bsnoc-ldata & straight line & $\underbrace{[L,\ldots,L]}_{\text{n-1 times}}$ & $\underbrace{[L,\ldots,L]}_{\text{n-1 times}}$ & $\underbrace{[L,\ldots,L]}_{\text{n-1 times}}$ & $\underbrace{[L,\ldots,L]]}_{\text{n-1 times}}$ \\ \hline
        6 & breverse & recursion & [1,$\ldots$,n] & [1,$\ldots$,n] & [n,$\ldots$,1] & [n,$\ldots$,1] \\ \hline
        7 & breverse-ldata & recursion & $\underbrace{[L,\ldots,L]}_{\text{n times}}$ & $\underbrace{[L,\ldots,L]}_{\text{n times}}$ & $\underbrace{[L,\ldots,L]}_{\text{n times}}$ & $\underbrace{[L,\ldots,L]}_{\text{n times}}$ \\ \hline
    \end{tabular*}
\end{table}

The first 5 cases merely use $n$ $\circ$ operators to make straight-line compositions from $n + 1$ similar \textit{pure} programs. We use the term \textit{pure} here to refer non-recursion compositions. The prefix \textit{lcomp} means that the associate precedence of $\circ$ in a straight-line composition is left. The suffix \textit{ldata} indicates that the input size is large. The symbol $L$ used in \textit{ldata} test cases can be expressed as a complex list as follows: $L = [T,\ldots,T] \ (n \ T\text{s}), \ T = [A,\ldots,A] \ (10 \ A\text{s}), \ A = [1,\ldots,5]$. These structures are only intended to generate data enough large for observing experimental results. 

We introduced the composition $phead \circ phead$ earlier. The compositions of many \textit{phead} do things similarly, the head of a head element inside a super nested list which expresses the source should be changed by the view. Because of the type of the source, this program is seen as a \textit{ldata} case. Next, we briefly explain the behavior of the remaining compositions in table \ref{tab:test-cases}.

$\putbx{ptail}{s}{v}$ accepts a source list $s$ and a view list $v$ to produce a new list by replacing the tail of $s$ with $v$. $\getbx{ptail}{s}$ return the tail of the source list $s$. The compositions of many $ptail$, in the \textit{put} direction, replace a part of the tail of the source list by the view list, in the \textit{get} direction, returns the such tail from the the source.

$\putbx{bsnoc}{s}{v}$ accepts a source $s$ and a view $v$ as two same length lists, to produce a new list by moving the last element of $v$ to its first position. $\getbx{ptail}{s}$ return another list by moving the first element of the list $s$ to its end position. The compositions of many \textit{ptail}, in the \textit{put} direction, create a permutation of the view list if the source list has a same length with the view, in the \textit{get} direction, produce a permutation of the source list.

\textit{breverse} is defined in terms of \textit{bfoldr}, in the \textit{put} direction, produces a reverse of the view list if the source list has a same length with the view, in the \textit{get} direction, produce a reverse of the source list.

\subsection{Results}

\newcommand{\qaddplot}[1]{
    \pgfplotstableread[col sep=comma]{csv/#1.csv}\data
    \addplot table[x=nComp,y=minbigul]{\data};
    \addplot table[x=nComp,y=pg]{\data};
    \addplot table[x=nComp,y=cpg]{\data};
    \addplot table[x=nComp,y=kpg]{\data};
    \addplot table[x=nComp,y=xpg]{\data};
}

\begin{figure}
    \centering
    \begin{tikzpicture}
        \pgfplotsset{footnotesize,samples=9}
        \begin{groupplot}[
                group style = {
                    group size = 2 by 5,
                    horizontal sep = 2.5cm,
                    vertical sep = 1cm,
                }, 
                width = 5cm, 
                height = 4cm,
                xmin = 0, xmax = 110000,
                ymin = 0,
                scaled x ticks = false,
                x tick label style = {
                    /pgf/number format/fixed
                },
                scaled y ticks = false,
                y tick label style = {
                    /pgf/number format/fixed,
                    /pgf/number format/precision=2,
                },
                every x tick scale label/.style={at={(rel axis cs:1,0)},anchor=south west,inner sep=1pt},
                every y tick scale label/.style={at={(rel axis cs:0.05,1.05)},anchor=south east,inner sep=1pt},
                max space between ticks=1000pt,
                try min ticks = 5,
                label style={font=\small},
                tick label style={font=\small} 
            ]

            \nextgroupplot[ 
                title = {\textit{lassoc-comp-replace}}, 
                legend style = {legend to name = legendgroup,},
                scaled x ticks = {base 10:-5},
                scaled y ticks = {base 10:-2},
            ]
                \pgfplotstableread[col sep=comma]{csv/time-lassoc-comp-replace.csv}\data
                \addplot table[x=nComp,y=minbigul]{\data}; \label{plots:minbigul};
                \addplot table[x=nComp,y=pg]{\data}; \label{plots:pg};
                \addplot table[x=nComp,y=cpg]{\data}; \label{plots:cpg};
                \addplot table[x=nComp,y=kpg]{\data}; \label{plots:kpg};
                \addplot table[x=nComp,y=xpg]{\data}; \label{plots:xpg};
                \coordinate (topl) at (rel axis cs:0,1);

            \nextgroupplot[ 
                title = {\textit{lassoc-comp-replace}}, 
                legend style = {legend to name = grouplegend,},
                scaled x ticks = {base 10:-5},
                scaled y ticks = {base 10:-2},
            ]
                \qaddplot{mem-lassoc-comp-replace};
                \coordinate (topr) at (rel axis cs:1,1);

            \nextgroupplot[
                title = {\textit{rassoc-comp-replace}},
                scaled x ticks = {base 10:-5},
                scaled y ticks = {base 10:1},
            ]
                \qaddplot{time-rassoc-comp-replace};

            \nextgroupplot[
                title = {\textit{rassoc-comp-replace}},
                scaled x ticks = {base 10:-5},
                scaled y ticks = {base 10:-1},
            ]
                \qaddplot{mem-rassoc-comp-replace};

            \nextgroupplot[
                title = {\textit{lassoc-comp-phead}},
                scaled x ticks = {base 10:-5},
                scaled y ticks = {base 10:-3},
                ylabel = {evaluation time (s)},
            ] 
                \qaddplot{time-lassoc-comp-phead};

            \nextgroupplot[
                title = {\textit{lassoc-comp-phead}},
                scaled x ticks = {base 10:-5},
                scaled y ticks = {base 10:-5},
                ylabel = {memory allocation (MBytes)},
            ] 
                \qaddplot{mem-lassoc-comp-phead};

            \nextgroupplot[
                title = {\textit{rassoc-comp-phead}},
                scaled x ticks = {base 10:-5},
                scaled y ticks = {base 10:-2},
            ] 
                \qaddplot{time-rassoc-comp-phead};

            \nextgroupplot[
                title = {\textit{rassoc-comp-phead}},
                scaled x ticks = {base 10:-5},
                scaled y ticks = {base 10:-2},
            ] 
                \qaddplot{mem-rassoc-comp-phead};

            \nextgroupplot[
                title = {\textit{breverse}},
                xmax = 10000,
                scaled x ticks = {base 10:-4},
                scaled y ticks = {base 10:-3},
                xlabel = {number of compositions},
            ] 
                \qaddplot{time-breverse};
                \coordinate (botl) at (rel axis cs:0,0);

            \nextgroupplot[
                title = {\textit{breverse}},
                xmax = 10000,
                scaled x ticks = {base 10:-4},
                scaled y ticks = {base 10:-5},
                xlabel = {number of compositions},
            ] 
                \qaddplot{mem-breverse};
                \coordinate (botr) at (rel axis cs:1,0);
        \end{groupplot}

        \path (topl|-current bounding box.north)--
            coordinate(legendpos)
            (botr|-current bounding box.north);
        \matrix[
            matrix of nodes,
            anchor = south,
            draw,
            inner sep = 0.2em,
            draw
        ]at([yshift=1ex]legendpos) {
            \ref{plots:minbigul} & minbigul & [2pt]
            \ref{plots:pg} & pg & [2pt]
            \ref{plots:cpg} & cpg & [2pt]
            \ref{plots:kpg} & kpg & [2pt]
            \ref{plots:xpg} & xpg\\
        };

        \path (botl|-current bounding box.south)--
            coordinate(subfig)
            (botr|-current bounding box.south);
        \matrix[
            matrix of nodes,
            anchor = east,
        ]at([xshift=-2.5cm,yshift=-2ex]subfig) {\textbf{(a)}\\};
        \matrix[
            matrix of nodes,
            anchor = west,
        ]at([xshift=2.5cm,yshift=-2ex]subfig) {\textbf{(b)}\\};

    \end{tikzpicture}

    \caption{Experimental results \quad \textbf{(a)} evaluation time \quad \textbf{(b)} memory allocation}
    \label{fig:result}

\end{figure}

Figure \ref{fig:result} illustrates the evaluation times of $put$ in minBiGUL, $put_m$ in minBiGUL$_m$ and $xpg$ for 7 test cases. We simply ignore the experimental results of $pg$, $cpg$ and $kpg$ since they are shown the slowness compared to $xpg$. For left associative compositions, $put$ works poorly because of the number of reevaluated $get$s. The figure shows $put_m$ and $xpg$ are better than $put$ in some cases. The left of figure contains tests using not very big input, and the right includes of the others. We see that $put_m$ is the fastest method if the input size is not too large. However, if the input size is large enough, $put_m$ will be slower quickly due to times for manipulating data in the table. At that time, $xpg$ proved to be the most effective method.


\section{Related Work}

\subsubsection*{BX languages and implementations}

There are many BX languages \cite{josh_pepm} \cite{BXtend} \cite{eMoflon} \cite{EVL+Strace} \cite{JTL} \cite{NMF}
%(memo: BXtend, eMoflon, EVL+Strace, JTL, NMF. SDMLib should be added?)
based on several application scenarios and BX lenses works \cite{} (add papers for lenses). The semantics and correctness of bidirectional programs has been investigated intensively during the past years, but their efficiency and optimization are not yet fully understood. Anjorin et. al. introduces the first benchmark for BX languages and compared them \cite{BXcomp}. However, the improvement for practical implementation of BX languages is still missing. As a first step, we study different evaluation methods to optimized their evaluation. We can say this paper will be first attempt to improve efficiency of BX composition evaluation. In this paper we focus on a BX language BiGUL \cite{josh_pepm} \cite{josh_popl}, we compare BiGUL's method (in Section 1) and our methods (in Section 3 and 4). From experimental results, we find that there is no best approach for all BX programs. It is important to choose good method depending on the BX programs and their inputs. In this paper we focus on BiGUL, however our methods can be applicable to other BX languages which have similar problems.
% its implementation uses ``not keeping any intermediate states and obtaining them by evaluation when they are needed'' strategy. Other BX languages might have the same problem, and our approach can be applicable.

\subsubsection*{Optimization techniques}

In this paper, we use several optimization techniques: tupling, lazy update, and lazy computation (\textcolor{red}{also fusion?}).

Tupling \cite{tupling} is an optimization technique by combining several computations. Although tupling is usually used for reducing the evaluation cost\cite{}, we use it to reduce the size of the things need to keep.

We introduce lazy update to remove repeated evaluation of put. This is different from lazy evaluation. In lazy evaluation we have everything that is needed to evaluate. In our lazy update, because view information is missing, we postpone the evaluation. (\textcolor{red}{Relation between delta lenses?})

Lazy evaluation \cite{} is well-known and widly-used optimization technique to reduce the redundant evaluation. Because we use OCaml for implementation, we need to introduce this explicitly. If we use Haskell, a lazy evaluated language, we do not need to introduce them.

%\begin{itemize}
    %\item BiGUL papers (PEPM, POPL, tutorial?)
    %\item the first BX paper
    %\item get-based BX
    %\item put-based BX?
%    \item implementation of BX, they might have the same problem, this paper will be first attempt, semantics and correctness is considered more, but efficiency is  lacked.
%    \item optimization techniques : tupling, lazy computation, fusion transformation?
%    \item (view update problem paper)
%\end{itemize}


\section{Conclusion and Future Work}

In this paper, we focus on efficiency of composition of BX programs.
To achieve fast evaluation, we introduce two different methods using memoization.
From the experimental results, we know that there is no best approach for all BX programs. However, if programmer can choose one method based on the BX programs and inputs, we can use better evaluation methods.

We will continue our work on the following points. First is extention of our language. Currently, because $pg$ requires very-well-behaved property



\begin{itemize}
\item overcome our limitations
  \begin{itemize}
  \item very-well-behaved $\to$ only xpg?
  \item In case expressions, path's problem $\to$ only xpg?
  \item How to obtain $\to$ only xpg?
  \end{itemize}
  \item tests for big BX programs
\item Type system \& Typing -- for safety
\end{itemize}



%The essential finding comes from the idea of tupling: in very-well behaved BX programs we can use $put$ as a compliment function for $get$, and vice versa.
%Based on the idea, we introduced $pg$, and improved it by several optimization techniques. From experimental results, our fastest approach $xpg$ is faster than other approaches for non purely right associative compositions. For right associative compositions, the original approach (miniBiGUL) is faster than $xpg$ because $xpg$ has some overhead cost. However this is not a problem, because usually programs are mix of left and right associative. If programmers know that their programs are purely right associative, they can choose miniBiGUL.

%We will continue our work on the following points. First, our target language is limited to very-well behaved, because our main idea requires the put-put property. However, for practical programs, we need to extend our work to overcome the current limitations.

%\begin{itemize}
%\item Extend our approach -- overcome our limitations
%  \begin{itemize}
%  \item treat well-behaved programs -- How to treat adaptive cases
%  \item In case expressions, the programs that use the different paths (put and get)
%  \end{itemize}
%\item How to obtain dummy?
%\item Type system \& Typing -- for safety
%\end{itemize}


\subsubsection*{Acknowledgment}
We thank the anonymous reviewers for valuable comments and suggestions. This work has been partially supported by JSPS KAKENHI Grant Number JP17H06099.

% From sample code
% 
% ---- Bibliography ----
%
% BibTeX users should specify bibliography style 'splncs04'.
% References will then be sorted and formatted in the correct style.
%
% \bibliography{bibliography}
% \bibliographystyle{splncs04}

\printbibliography

\end{document}
