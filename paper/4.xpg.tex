\section{Tupling and Lazy Updates: xpg} \label{sect:xpg}
\subsection{Tupling: pg}

Another solution for saving intermediate states is tupling. If $put$ and $get$ are evaluated simultaneously, it is potential to reduce the number of recomputed $get$s. The following function, $pg$, accepts the pair of a source and a view as the input to produce a new pair that contains the actual result of the corresponding minBiGUL program.

\begin{definition}
    $\pg{bx}{s}{v} = (\putbx{bx}{s}{v}, \getbx{bx}{s})$
\end{definition}

% $pg$ is an involution that is self-inverse. An involution is a function $f$ that satisfies $f(f(x)) = x \text{ for all } x \text{ in the domain of } f$.

% \begin{proof}
% $pg [\![bx]\!] \ (\pg{bx}{s}{v}) \\
%     \tab = \pg{bx}{(\putbx{bx}{s}{v})}{(\getbx{bx}{s})} \quad [pg \text{ definition}] \\
%     \tab = (put \ [\![bx]\!] \ (\putbx{bx}{s}{v}) \ (\getbx{bx}{s}), \getbx{bx}{(\putbx{bx}{s}{v})})  \quad [pg \text{ definition}] \\
%     \tab = (put \ [\![bx]\!] \ (\putbx{bx}{s}{v}, \getbx{bx}{s}), v) \quad [\textsc{PutGet}] \\
%     \tab = (put \ [\![bx]\!] \ (s, \getbx{bx}{s}), v) \quad [\textsc{PutPut}] \\
%     \tab = (s, v) \quad [\textsc{GetPut}]$
% \end{proof}

\noindent Now, let's see how we construct $pg$ recursively.

    \noindent $\tabs{2} \pg{Skip \ h}{s}{v} \overset{1}{=} (\sif{h \ s = v}{s}{\text{fail}}, h \ s) \\
        \tab \qtab \overset{2}{=} \sif{h \ s = v}{(s, h \ s)}{\text{fail}} \\
        \tab \qtab \overset{3}{=} \sif{h \ s = v}{(s, v)}{\text{fail}}$

The first equality is simply based on the definitions of $pg$, $\putbxinline{Skip \ h}$ and $\getbxinline{Skip \ h}$. The second one tuples two results of $put$ and $get$ in the body of the if-expression. This is a trick since in some cases, the result of $pg$ may be fail although there is no fail when evaluating $\getbxinline{Skip \ h}$. The last equality is relatively oblivious.

    \noindent $\tabs{2} \pg{Replace}{s}{v} = (v, s)$\\
    $\tabs{2} \pg{\product{bx_1}{bx_2}}{(s_1,s_2)}{(v_1,v_2)}\\
        \tab \qtab \overset{1}{=} ((\putbx{bx_1}{s_1}{v_1}),(\putbx{bx_2}{s_2}{v_2}), (\getbx{bx_1}{s_1}), (\getbx{bx_2}{s_2})) \\
        \tab \qtab \overset{2}{=} (s_1, v_1) \Leftarrow \pg{bx_1}{s_1}{v_1}; \\
            \qtab \tabs{2} \ \, (s_2, v_2) \Leftarrow \pg{bx_2}{s_2}{v_2}; \\
            \tab \qtabs{2} ((s_1,s_2), (v_1,v_2))$\\
    $\tabs{2} \pg{\rearrs{f_1}{f_2}{bx}}{s}{v} \overset{1}{=} (f_2 \ (\putbx{bx}{(f_1 \ s)}{v}), \getbx{bx}{(f_1 \ s)}) \\
        \tab \qtab \overset{2}{=} (s, v) \Leftarrow \pg{bx}{f_1 \ s}{v};\\
            \tab \qtabs{2} (f_2 \ s, v)$\\
    $\tabs{2} \pg{\rearrv{g_1}{g_2}{bx}}{s}{v} \overset{1}{=} (\putbx{bx}{s}{(g_1 \ v)}, g_2 \ (\getbx{bx}{s})) \\
        \tab \qtab \overset{2}{=} (s, v) \Leftarrow \pg{bx}{s}{g_1 \ v}; \\
            \tab \qtabs{2} (s, g_2 \ v)$

Constructions of $pg$ for the replacement, the product and the source/view rearrangements are very clear when just paring $put$ and $get$ respectively, then doing basic changes.

    \noindent $\tabs{2} \pg{\casebx{cond_{sv}}{cond_{s}}{bx_1}{bx_2}}{s}{v} \\
    \tabs{3} \overset{1}{=} (\text{if} \ {cond_{sv} \ s \ v} \qtabs{3} \tab \text{if} \ {cond_{s} \ s} \\
    \tabs{2} \qtab \text{then} \ s' \Leftarrow \putbx{bx_1}{s}{v} \qtab \tab \, \text{then} \ v' \Leftarrow \getbx{bx_1}{s} \\
    \tabs{2} \qtab \text{else} \ s' \Leftarrow \putbx{bx_2}{s}{v} \qtab \tab \ \, \text{else} \ v' \Leftarrow \getbx{bx_2}{s}\\
    \tabs{2} \qtab \text{fi} \ cond_{s} \ s'; \text{ return } s' \qtab , \tab \ \ \text{fi} \ {cond_{sv} \ s \ v'}; \text{ return } v')\\
    \tabs{3} \overset{2}{=} \text{if} \ {cond_{sv} \ s \ v} \ \&\& \ {cond_{s} \ s}\\
    \tabs{2} \qtab \text{then} \ (s', v') \Leftarrow \pg{bx_1}{s}{v}\\
    \tabs{2} \qtab \text{else} \ (s', v') \Leftarrow \pg{bx_2}{s}{v}\\
    \tabs{2} \qtab \text{fi} \ cond_{s} \ s' \ \&\& \ cond_{sv} \ s \ v'; \text{ return } (s',v')$

A restriction for $\pginline{Case}$ needs to be introduced here. We know that there is one entering condition and one exit condition when evaluating $\putbxinline{Case}$ as well as $\getbxinline{Case}$. If a tupling occurs, there will be 4 combinations from these conditions. This means two entering conditions of $\putbxinline{Case}$ and $\getbxinline{Case}$ are not always simultaneously satisfied. The evaluated branches are distinct in the $put$ and $get$ directions for combinations $((cond_{sv} \ s \ v) \ \&\& \ (not (cond_{s} \ s)))$ and $((not (cond_{sv} \ s \ v) \ \&\& \ (cond_{s} \ s))$, which are restricted in this paper. This does not happen for the others which is used in the construction of $\pginline{Case}$.

    \noindent $\tabs{2} \pg{bx_1 \circ bx_2}{s}{v} \\
    \tabs{3} \overset{1}{=} (\putbx{bx_1}{s}{(\putbx{bx_2}{(\getbx{bx_1}{s})}{v})}, \getbx{bx_2}{(\getbx{bx_1}{s})}) \\
    \tabs{3} \overset{2}{=} v_1 \Leftarrow \getbx{bx_1}{s}; \qtabs{3} \ \, \overset{3}{=} (s_1, v_1) \Leftarrow \pg{bx_1}{s}{dummy};\\
        \tab \qtab (s_2, v_2) \Leftarrow \pg{bx_2}{v_1}{v}; \qtabs{2} (s_2, v_2) \Leftarrow \pg{bx_2}{v_1}{v};\\
        \tab \qtab (s_3, v_3) \Leftarrow \pg{bx_1}{s}{s_2}; \qtabs{2} (s_3, v_3) \Leftarrow \pg{bx_1}{s}{s_2};\\
            \qtab \tabs{2} (s_3, v_2) \qtabs{6} (s_3, v_2)\\
    \tabs{3} \overset{4}{=} (s_1, v_1) \Leftarrow \pg{bx_1}{s}{construct\_dummy \ s}; \\
        \tab \qtab (s_2, v_2) \Leftarrow \pg{bx_2}{v_1}{v}; \\
        \tab \qtab (s_3, v_3) \Leftarrow \pg{bx_1}{s_1}{s_2}; \\
            \qtab \tabs{2} (s_3, v_2)$

The construction of $\pginline{bx_1 \circ bx_2}$ can be considered as the soul of the $pg$ function. The first two equalities comes from mentioned definitions and some basic transformations. The third one rewrites $v_1 \Leftarrow \getbx{bx_1}{s}$ into $(s_1, v_1) \Leftarrow \pg{bx_1}{s}{dummy}$. This is possible when we consider $\getbx{bx_1}{s}$ as the second element of $\pg{bx_1}{s}{dummy}$ where $dummy$ is a special value that makes the $put \, [\![bx_1]\!]$ valid. We can construct a $dummy$ from the source by \textsc{GetPut} law in some cases or require a valid view from programmers. Since there is no real view, these things are necessary. The last equality changes $dummy$ by an application $construct\_dummy \ s$, and also substitutes $s$ in $(s_3, v_3) \Leftarrow \pg{bx_1}{s}{s_2}$ by $s_1$ which equals to $\putbx{bx_1}{s}{(construct\_dummy \ s)}$. The replacement of $s$ and $s_1$ is possible because of the \textsc{PutPut} law. Then, $v_3$ in the result pair $(s_3,v_3)$ equals $dummy$. So we can realize that there is no loss information when computing a composition.

\subsection{Lazy Update: cpg}

When evaluating $\pginline{bx_1 \circ bx_2}$, there are three $pg$ calls, of which twice for $\pginline{bx_1}$ and once for $\pginline{bx_2}$. If a given program is a left-associative composition, the number of $pg$ calls will be exponential. Therefore, the runtime inefficiency is inevitable. To solve that, we introduce a new function, $cpg$, accumulates changes in the source and the view. $\cpg{bx}{ks}{kv}{s}{v}$ is an extension of $\pg{bx}{s}{v}$ where $ks$ and $kv$ are continuations used to hold the modification information, and $s$ and $v$ are used to keep evaluated values. The output of this function is a 4-tuple $(ks, kv, s, v)$. To be more convenient for presenting the definition of $cpg$ as well as the other functions later, we provide some following utility functions:\\ $fst = \lambda (x_1,x_2). x_1 , \ snd = \lambda (x_1,x_2). x_2 , \ con = \lambda ks_1. \lambda ks_2. \lambda x. ((ks_1 \ x),(ks_2 \ x))$

\begin{definition}
$\cpg{bx}{ks}{kv}{s}{v}$

    \noindent $\cpg{Skip \ h}{ks}{kv}{s}{v} = \text{ if } h \ s = v \text{ then } (ks, kv, s, v) \text{ else fail}$

    \noindent $\cpg{Replace}{ks}{kv}{s}{v} = (kv, ks, v, s)$

    \noindent $\cpg{\product{bx_1}{bx_2}}{ks}{kv}{s}{v} =\\
        \tab (ks_1, kv_1, s_1, v_1) \Leftarrow \cpg{bx_1}{fst \circ ks}{fst \circ kv}{fst \ s}{fst \ v};\\
        \tab (ks_2, kv_2, s_2, v_2) \Leftarrow \cpg{bx_2}{snd \circ ks}{snd \circ kv}{snd \ s}{snd \ v};\\
        \qtab (con \ ks_1 \ ks_2, con \ kv_1 \ kv_2, (s_1,s_2), (v_1,v_2))$

    \noindent $\cpg{RearrS \ f_1 \ f_2 \ bx}{ks}{kv}{s}{v} =\\
        \tab (ks, kv, s, v) \Leftarrow \cpg{bx}{f_1 \circ ks}{kv}{f_1 \ s}{v};\\
        \qtab (f_2 \circ ks, kv, s, v)$

    \noindent $\cpg{RearrV \ g_1 \ g_2 \ bx}{ks}{kv}{s}{v} =\\
        \tab (ks, kv, s, v) \Leftarrow \cpg{bx}{ks}{g_1 \circ kv}{s}{g_1 \ v};\\
        \qtab (ks, g_2 \circ kv, ks, g_2 \ v)$

    \noindent $\cpg{Case \ cond_{sv} \ cond_{s} \ bx_1 \ bx_2}{ks}{kv}{s}{v} =\\
        \tab \text{if} \ cond_{sv} \ s \ v \ \&\& \ cond_{s} \ s\\
        \tab \text{then} \ (ks, kv, s', v') \Leftarrow \cpg{bx_1}{ks}{kv}{s}{v}\\
        \tab \text{else} \ (ks, kv, s', v') \Leftarrow \cpg{bx_2}{ks}{kv}{s}{v}\\
        \tab \text{fi} \ cond_{s} \ s' \ \&\& \ cond_{sv} \ s \ v'; \ \text{return} \ (ks, kv, s', v')$

    \noindent $\cpg{bx_1 \circ bx_2}{ks}{kv}{s}{v} =\\
        \tab (ks_1, kv_1, s_1, v_1) \Leftarrow \cpg{bx_1}{ks}{id}{s}{construct\_dummy \ s};\\
        \tab (ks_2, kv_2, s_2, v_2) \Leftarrow \cpg{bx_2}{kv_1}{kv}{v_1}{v};\\
            \qtab (ks_1 \circ ks_2, kv_2,  ks_1 \ s_2, v_2)$
\end{definition}

Apart from the last construction, the others are quite similar to the corresponding ones of $pg$, but have some updates over the source and the view on accumulative functions $ks$ and $kv$ respectively.

For $\cpginline{bx_1 \circ bx_2}$, there are only two $cpg$ calls. The first call $\cpginline{bx_1}$ requires parameter $(ks, id, s, construct\_dummy \ s)$ where $s$ and $ks$ are corresponding to the source and the update over source. Since there is no real view here, we need to construct a dummy from the source. Then the continuation updating on this dummy should be initiated as an identity function. The first $cpg$ call is assigned to a 4-tuple $(ks_1, kv_1, s_1, v_1)$ where $s_1$ is redundant because this evaluated variable is not used in the later steps. In the next assignment, a 4-tuple $(ks_2, kv_2, s_2, v_2)$ is assigned by the second $cpg$ call which uses the input as $(kv_1, kv, v_1, v)$ where $kv_1$ and $v_1$ are obtained from the result of the first assignment, and $kv$ and $v$ come from the input. It is relatively similar to the second $pg$ call assignment in $\pginline{bx_1 \circ bx_2}$. After two $cpg$ calls, a function application, $ks_1 \ s_2$, is used to produce the updated source instead of calling recursively one more time like in $\pginline{bx_1 \circ bx_2}$.

Suppose that we have a source $s_0$ and a view $v_0$. The pair of the updated source and view $(s, v)$ where $s = \putbx{bx}{s_0}{v_0}$ and $v = \getbx{bx}{s_0}$ can be obtained using $cpg$ as follows:

    $\tab s \Leftarrow s_0; v \Leftarrow v_0; (ks, kv, s, v) \Leftarrow \cpg{bx}{\lambda \_.s}{id}{s}{v};\\
        \tab \qtab (s; v)$

In general, the beginning of a continuation should be an identity function. However, to be able to use the function application to get the result of $\cpginline{bx_1 \circ bx_2}$, the accumulative function on the source $s$ needs to be initiated as $\lambda \_.s$. This constant function helps to retain the discarded things in the source.

% Suppose the beginning of continuations $ks$ and $kv$ are $ks_0$ and $id$ respectively. Let's consider $\cpg{phead \circ phead}{ks_0}{id}{s}{v}$ where $s = [[1,2,3], [\,], [4,5]]$ and $v = 100$. After the first two assignments in the definition of $cpg$ for the composition, we have: $ks_1 = f_2 \circ (con \ (fst \circ g_1 \circ id) \ (snd \circ f_1 \circ ks_0))$ and $s_2 = [100,2,3]$ where $f_1 = \lambda (s::ss).(s,ss)$, $f_2 = \lambda (s,ss).(s::ss)$, $g_1 = \lambda v.(v,())$. Then:

% $ks_1 \ s_2 = (f_2 \circ (con \ (fst \circ g_1 \circ id) \ (snd \circ f_1 \circ ks_0))) \ s_2 \\
% \tab \quad = f_2 \ (\ (fst(g_1(id(s_2))) \ , \ snd(f_1(ks_0(s_2))))\ )\\
% \tab \quad = fst(g_1(id(s_2))) :: snd(f_1(ks_0(s_2))) = [100,2,3] :: snd(f_1(ks_0([100,2,3])))$

% If $ks_0$ is an identity function, $ks_1 \ s_2 = [100,2,3] :: [2,3]$. This is an unexpected result when we target it to be the updated source. If $ks_0 = \lambda \_.s$ where $s = [[1,2,3], [\,], [4,5]]$, the result will be what we want to see.$ks_1 \ s_2 = [100,2,3] :: [[\,], [4,5]] = [[100,2,3], [\,], [4,5]]$. Through the above example, using the continuation $ks$ as a constant function at the beginning contributes to retain the discarded things in the source.


\subsection{Lazy Computation: kpg}

The problem for $cpg$ lies in redundant computation during evaluating. For keeping away such computations, we introduce an extension named $kpg$. While $cpg$ evaluate values eagerly, $kpg$ does the opposite. Every values are evaluated lazily in a computation of $kpg$. The input of $\kpginline{bx}$ is expanded to a 6-tuple $(ks, kv, ks', kv', s, v)$ where $ks$ and $kv$ keep the modification information, $s$ and $v$ hold evaluated values, and $ks'$ and $kv'$ are used for lazy evaluation of actual values. The output of this function is also a 6-tuple $(ks, kv, ks', kv', s, v)$.\\
Suppose that we have a source $s_0$ and a view $v_0$. The pair of the updated source and view $(s, v)$ where $s = \putbx{bx}{s_0}{v_0}$ and $v = \getbx{bx}{s_0}$ can be obtained using $kpg$ as follows:

    $\tab s \Leftarrow s_0; v \Leftarrow v_0; (ks, kv, ks', kv', s, v) \Leftarrow \kpg{bx}{\lambda \_.s}{id}{id}{id}{s}{v};\\
        \tab \qtab (ks' \ s; kv' \ v)$

The beginning of accumulative functions $ks'$ and $kv'$ are set as identity, while $ks$ and $kv$ are initiated as the same with the corresponding ones in $cpg$.

\begin{definition}
$\kpg{bx}{ks}{kv}{ks'}{kv'}{s}{v}$

    \noindent $\kpg{Skip \ h}{ks}{kv}{ks'}{kv'}{s}{v} =\\
        \tab s \Leftarrow ks' \ s; \quad v \Leftarrow kv' \ v; \quad ks' \Leftarrow id; \quad kv' \Leftarrow id;\\
        \tab \text{if} \ h \ s = v \ \text{then} \ (ks, kv, ks', kv', s, v) \ \text{else fail}$

    \noindent $\kpg{Replace}{ks}{kv}{ks'}{kv'}{s}{v} = (kv, ks, kv', ks', v, s)$

    \noindent $\kpg{\product{bx_1}{bx_2}}{ks}{kv}{ks'}{kv'}{s}{v} =\\
        \tab s \Leftarrow ks' \ s; \quad v \Leftarrow kv' \ v; \quad ks' \Leftarrow id; \quad kv' \Leftarrow id;\\
        \tab (ks_1, kv_1, ks_1', kv_1', s_1, v_1) \Leftarrow \kpg{bx_1}{fst \circ ks}{fst \circ kv}{fst \circ ks'}{fst \circ kv'}{s}{v};\\
        \tab (ks_2, kv_2, ks_2', kv_2', s_2, v_2) \Leftarrow \kpg{bx_2}{snd \circ ks}{snd \circ kv}{snd \circ ks'}{snd \circ kv'}{s}{v};\\
        \qtab ( con \ ks_1 \ ks_2, con \ kv_1 \ kv_2, con \ (ks_1' \circ fst) \ (ks_2' \circ snd),\\
        \qtab con \ (kv_1' \circ fst) \ (kv_2' \circ snd), \\
        \qtab (s_1, s_2), (v_1,v_2))$

    \noindent $\kpg{RearrS \ f_1 \ f_2 \ bx}{ks}{kv}{ks'}{kv'}{s}{v} =\\
        \tab (ks, kv, ks', kv', s, v) \Leftarrow \kpg{bx}{f_1 \circ ks}{kv}{f_1 \circ ks'}{kv'}{s}{v};\\
        \qtab (f_2 \circ ks, kv, f_2 \circ ks', kv', s, v)$

    \noindent $\kpg{RearrV \ g_1 \ g_2 \ bx}{ks}{kv}{ks'}{kv'}{s}{v} =\\
        \tab (ks, kv, ks', kv', s, v) \Leftarrow \kpg{bx}{ks}{g_1 \circ kv}{ks'}{g_1 \circ kv'}{s}{v};\\
        \qtab (ks, g_2 \circ kv, ks', g_2 \circ kv', s, v)$

    \noindent $\kpg{Case \ cond_{sv} \ cond_{s} \ bx_1 \ bx_2}{ks}{kv}{ks'}{kv'}{s}{v} =\\
        \tab s \Leftarrow ks' \ s; \quad v \Leftarrow kv' \ v; \quad ks' \Leftarrow id ; \quad kv' \Leftarrow id;\\
        \tab \text{if} \ cond_{sv} \ s \ v \ \&\& \ cond_{s} \ s\\
        \tab \text{then} \ (ks, kv, ks', kv', s', v') \Leftarrow \kpg{bx_1}{ks}{kv}{ks'}{kv'}{s}{v}\\
        \tab \text{else} \ (ks, kv, ks', kv', s', v') \Leftarrow \kpg{bx_2}{ks}{kv}{ks'}{kv'}{s}{v}\\
        \tab \text{fi} \ cond_s \ (ks' \ s') \ \&\& \ cond_{sv} \ s \ (kv' \ v'); \ \text{return} \ (ks, kv, ks', kv', s', v')$

    \noindent $\kpg{bx_1 \circ bx_2}{ks}{kv}{ks'}{kv'}{s}{v} =\\
        \tab (ks_1, kv_1, ks_1', kv_1', s_1, v_1) \Leftarrow \kpg{bx_1}{ks}{id}{ks'}{id}{s}{construct\_dummy \ s};\\
        \tab (ks_2, kv_2, ks_2', kv_2', s_2, v_2) \Leftarrow \kpg{bx_2}{kv_1}{kv}{kv_1'}{kv'}{v_1}{v};\\
        \qtab (ks_1 \circ ks_2, kv_2, ks_1 \circ ks_2', kv_2', s_2, v_2)$
\end{definition}

Except $\kpginline{Skip}$ and $\kpginline{Case}$ where $s$ and $v$ hold actual values, functions for the computation will be kept in $ks'$ and $kv'$. When evaluating $\kpginline{bx_1 \circ bx_2}$, $ks_1'$ and $s_1$ in the result of the first assignment are still redundant but they are not evaluated. In $\kpginline{bx_1 \times bx_2}$, the evaluation of $ks'$ and $kv'$ will be done independently in two assignments using $\kpginline{bx_1}$ and $\kpginline{bx_2}$. There may be same computations in $fst \circ ks'$ and $snd \circ ks'$ as well as $fst \circ kv'$ and $snd \circ kv'$. It is possible to evaluate actual values in $s$ and $v$ before calling $\kpginline{bx_1}$ to remove the redundancy like that. One more thing, in $\kpginline{Case}$, we need to evaluate $ks' \ s'$ and $kv' \ v'$ to check the fi condition before returning the 6-tuple. Such evaluations can be done lazily to make programs run faster. We use the above small optimizations in our implementation.

\subsection{Combination of pg and kpg: xpg}
The most important thing when using $kpg$ to evaluate a composition $bx_1 \circ bx_2$ is keeping the dropped parts from the source in a function that is used to produce new source instead of using more recursive call. This is reflected in the first call $\kpginline{bx_1}$ and the last application in $\kpginline{bx_1 \circ bx_2}$. The second call $\kpginline{bx_2}$ just play a syntactic role while it requires many parameters. We can completely reduce this number to 2, including one for the source and one for the view as in a $pg$ call. This idea is realized by function $xpg$ as follows:

\begin{definition}
$\xpg{bx}{s}{v}$

    \noindent $\xpg{bx}{s}{v} = \text{match } bx \text{ with }\\
    \tab \vert \ bx_1 \circ bx_2 \to \\
        \tabs{2} (ks_1,kv_1, ks_1', kv_1', s_1, v_1) \Leftarrow \kpg{bx_1}{\lambda \_.s}{id}{id}{id}{s}{construct\_dummy \ s};\\
        \tabs{2} (s_2, v_2) \Leftarrow \xpg{bx_2}{kv_1' \ v_1}{v};\\
        \tabs{3} (ks_1 \ s_2, v_2)\\
    \tab \vert \ \_ \to \text{similar to } put$
\end{definition}

Similar to $pg$, $\xpginline{bx}$ accepts a pair of the source and the view $(s,v)$ to produce the new pair. The constructions of $\xpginline{bx}$ when $bx$ is not a composition are the same as the ones of $\pginline{bx}$. Note that, $xpg$ is called recursively instead of $pg$. For $\xpginline{bx_1 \circ bx_2}$, we use two function calls and a function application to calculate the result. The first call and the function application come from $kpg$, while the second call is based on $pg$.
