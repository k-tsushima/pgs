\section{Experiments} \label{sect:experiments}

We have fully implemented and tested all methods% 
%
\footnote{All experiments on macOS 10.14.6, processor Intel Core i7 (2.6~GHz), RAM 16~GB 2400~MHz DDR4, OCaml 4.07.1. The OCaml runtime system options and garbage collection parameters are set as default. Our target language is untyped.}$^,$%HACK!
%
\footnote{Anonymous \url{https://mega.nz/\#!yUsgyCxA!yrJfINiwtCwhmd4McYMz2xsFa-ljkgGKmMi839ut5gA}}
%
described in the previous sections.
Instead of constructing some dummies by $construct\_dummy$s, which are necessary for $pg$, $cpg$, $kpg$ and $xpg$, we replace them by the current updated views. This helps a program in the $put$ direction valid.

\subsection{Test Cases}

We have 
% conducted 
selected seven test cases (Table~\ref{tab:test-cases}) to 
represent
% assess 
non-trivial cases of practical significance.
The test cases use
% These $bx$s are 
left associative compositions because we focus on this kind of inefficiency in this paper.
% Although the number of cases is small, 
%% they are not trivial and can be completely used in practice. 
% they represent non-trivial cases of practical significance.
In the last two columns, $s_r$ and $v_r$ are the updated source and view, respectively. They are produced by applying $put$ and $get$ to the original source $s_0$ and view $v_0$. That is, $s_r = \putbx{bx}{s_0}{v_0}$ and $v_r = \getbx{bx}{s_0}$, where $bx$ is the program indicated in the second column of the table. 
% are only interested in this kind of programs in the paper. 
% Please note that the results
Results $s_r$ and $v_r$ are independent of the associativity 
% style 
of the composition.

\begin{table}[hbt!]
    \centering
    \caption{Composition test cases}
    \label{tab:test-cases}
    \begin{tabular*}{\textwidth}{|l @{\extracolsep{\fill}}|l|l|c|c|c|c|}
        \hline
        \multirow{2}{*}{No} & \multicolumn{1}{c|}{\multirow{2}{*}{Name}} & \multicolumn{1}{c|}{\multirow{2}{*}{Type}} & \multicolumn{2}{c|}{Input} & \multicolumn{2}{c|}{Output} \\ \cline{4-7} 
        & \multicolumn{1}{c|}{} & \multicolumn{1}{c|}{} & \multicolumn{1}{c|}{$s_0$} & \multicolumn{1}{c|}{$v_0$} & \multicolumn{1}{c|}{$s_r$} & \multicolumn{1}{c|}{$v_r$} \\ \hline
        1 & lcomp-phead-ldata & straight line & $\underbrace{[[\ldots[1]\ldots]]}_{\text{n+1 times}}$ & 100 & $\underbrace{[[\ldots[100]\ldots]]}_{\text{n+1 times}}$ & 1 \\ \hline
        2 & lcomp-ptail & straight line & [1,$\ldots$,n+1] & [1,$\ldots$,10] & $s_0 \ @ \ v_0$ & [\ ] \\ \hline
        3 & lcomp-ptail-ldata & straight line & $\underbrace{[L,\ldots,L]}_{\text{n+1 times}}$ & $\underbrace{[L,\ldots,L]}_{\text{10 times}}$ & $\underbrace{[L,\ldots,L]}_{\text{n+11 times}}$ & [\ ] \\ \hline
        4 & lcomp-bsnoc & straight line & [1,$\ldots$,n-1] & [1,$\ldots$,n-1] & [1,$\ldots$,n-1] & [1,$\ldots$,n-1] \\ \hline
        5 & lcomp-bsnoc-ldata & straight line & $\underbrace{[L,\ldots,L]}_{\text{n-1 times}}$ & $\underbrace{[L,\ldots,L]}_{\text{n-1 times}}$ & $\underbrace{[L,\ldots,L]}_{\text{n-1 times}}$ & $\underbrace{[L,\ldots,L]]}_{\text{n-1 times}}$ \\ \hline
        6 & breverse & recursion & [1,$\ldots$,n] & [1,$\ldots$,n] & [n,$\ldots$,1] & [n,$\ldots$,1] \\ \hline
        7 & breverse-ldata & recursion & $\underbrace{[L,\ldots,L]}_{\text{n times}}$ & $\underbrace{[L,\ldots,L]}_{\text{n times}}$ & $\underbrace{[L,\ldots,L]}_{\text{n times}}$ & $\underbrace{[L,\ldots,L]}_{\text{n times}}$ \\ \hline
    \end{tabular*}
\end{table}

% The first 5 cases merely use $n$ $\circ$ operators to make straight-line compositions from the same $n + 1$ \textit{pure} programs. 
The first five test cases
(1--5) 
are $n$ straight-line (non-recursive) compositions of the same $n + 1$ 
% non-recursive 
programs.
% We refer to them as `pure' compositions.
% We use the term \textit{pure} here to refer non-recursion compositions. 
The prefix \textit{lcomp} in the name of a test case indicates that the textual compositions are left associative.
% that the associate precedence of $\circ$ in a straight-line composition is left.
The suffix \textit{ldata} indicates that the input size is considered large. The symbol $L$ in the input column 
% \textit{ldata} test cases 
% can be expressed as a complex list as follows: 
stands for a list $L = [T,\ldots,T]$ with
% $(n \ T\text{s}), 
$T = [A,\ldots,A]$ of length 10
% \ (10 \ A\text{s}),
and
$A = [1,\ldots,5]$. 
% These structures are 
They are only intended to generate test data that is large enough for 
% observing 
measuring
% experimental 
results. 

We introduced the composition $phead \circ phead$ earlier in Section~\ref{sec:intro}. The composition of many \textit{phead} 
% do things
works similarly. The head of a head element inside a 
%super 
deeply nested list, which 
% expresses 
is the source, 
% should be 
is updated by the changed view. Because of the type of the source, this program is 
% seen 
categorized as a \textit{ldata} case. 

Next, we briefly explain the behavior of the remaining compositions in 
\mbox{Table\,\ref{tab:test-cases}.}%HACK!

$\putbx{ptail}{s}{v}$ accepts a source list $s$ and a view list $v$ to produce a new list by replacing the tail of $s$ with $v$. $\getbx{ptail}{s}$ returns the tail of the source list $s$. The composition of many $ptail$, in the \textit{put} direction, replaces a part of the tail of the source list by the view list and, in the \textit{get} direction, returns such a tail from the source.

$\putbx{bsnoc}{s}{v}$ accepts a source $s$ and a view $v$, 
%as two same length lists, 
which are two lists of the same length,
and produces a new list by moving the last element of $v$ to its first position. $\getbx{ptail}{s}$ returns another list by moving the first element of the list $s$ to its end position. The composition of many \textit{ptail}, in the \textit{put} direction, creates a permutation of the view list if the source list has a same length as the view and, in the \textit{get} direction, produces a permutation of the source list.

\textit{breverse} is defined in terms of \textit{bfoldr}. In the \textit{put} direction, it produces a reverse of the view list if the source list has a same length as the view and, in the \textit{get} direction, produces a reverse of the source list. Note that compositions are by the recursions of $breverse$ and 
the number of compositions are dynamically determined by the length of the source list.

\subsection{Results}

\newcommand{\qaddplot}[1]{
    \pgfplotstableread[col sep=comma]{csv/#1.csv}\data
    \addplot table[x=nComp,y=minbigul]{\data};
    \addplot table[x=nComp,y=minbigul_m]{\data};
    \addplot table[x=nComp,y=xpg]{\data};
}

\begin{figure}
    \centering
    \begin{tikzpicture}
        \pgfplotsset{footnotesize,samples=9}
        \begin{groupplot}[
                group style = {
                    group size = 2 by 7,
                    horizontal sep = 2.5cm,
                    vertical sep = 1cm,
                }, 
                width = 4.5cm, 
                height = 3.5cm,
                % xmin = 0, xmax = 110000,
                % ymin = 0,
                scaled x ticks = false,
                x tick label style = {
                    /pgf/number format/fixed
                },
                scaled y ticks = false,
                y tick label style = {
                    /pgf/number format/fixed,
                    /pgf/number format/precision=2,
                },
                every x tick scale label/.style={at={(rel axis cs:1,0)},anchor=south west,inner sep=1pt},
                every y tick scale label/.style={at={(rel axis cs:-0.1,1.05)},anchor=south east,inner sep=1pt},
                max space between ticks=1000pt,
                try min ticks = 5,
                label style={font=\small},
                tick label style={font=\small} 
            ]

            \nextgroupplot[ 
                title = {\textit{lcomp-replace}}, 
                legend style = {legend to name = legendgroup,},
                scaled x ticks = {base 10:-5},
                scaled y ticks = {base 10:-1},
            ]
                \pgfplotstableread[col sep=comma]{csv/lcomp-replace.csv}\data
                \addplot table[x=nComp,y=minbigul]{\data}; \label{plots:put};
                \addplot table[x=nComp,y=minbigul_m]{\data}; \label{plots:put$_m$};
                \addplot table[x=nComp,y=xpg]{\data}; \label{plots:xpg};
                \coordinate (topl) at (rel axis cs:0,1);
            
            \nextgroupplot[ 
                title = {\textit{lcomp-replace-ldata}}, 
                legend style = {legend to name = grouplegend,},
                scaled x ticks = {base 10:-5},
                scaled y ticks = {base 10:-1},
            ]
                \qaddplot{lcomp-replace-ldata};
            
            \nextgroupplot [
                hide axis, xtick=\empty, ytick=\empty
            ]

            \nextgroupplot[ 
                title = {\textit{lcomp-phead-ldata}}, 
                scaled x ticks = {base 10:-5},
                scaled y ticks = {base 10:-2},
            ]
                \qaddplot{lcomp-phead-ldata};

            \nextgroupplot[
                title = {\textit{lcomp-ptail}},
                scaled x ticks = {base 10:-5},
                scaled y ticks = {base 10:-2},
            ]
                \qaddplot{lcomp-ptail};

            \nextgroupplot[
                title = {\textit{lcomp-ptail-ldata}},
                scaled x ticks = {base 10:-5},
                scaled y ticks = {base 10:-2},
            ] 
                \qaddplot{lcomp-ptail-ldata};

            \nextgroupplot[
                title = {\textit{lcomp-bsnoc}},
                scaled x ticks = {base 10:-4},
                scaled y ticks = {base 10:-2},
            ]
                \qaddplot{lcomp-bsnoc};

            \nextgroupplot[
                title = {\textit{lcomp-bsnoc-ldata}},
                scaled x ticks = {base 10:-4},
                scaled y ticks = {base 10:-2},
            ] 
                \qaddplot{lcomp-bsnoc-ldata};

            \nextgroupplot[
                title = {\textit{bmapreplace}},
                scaled x ticks = {base 10:-4},
                scaled y ticks = {base 10:-1},
            ]
                \qaddplot{bmapreplace};

            \nextgroupplot[
                title = {\textit{bmapreplace-ldata}},
                scaled x ticks = {base 10:-4},
                scaled y ticks = {base 10:-1},
            ] 
                \qaddplot{bmapreplace-ldata};

            \nextgroupplot[
                title = {\textit{breverse}},
                % xmax = 10000,
                scaled x ticks = {base 10:-5},
                scaled y ticks = {base 10:-3},
            ] 
                \qaddplot{breverse};
                % \coordinate (botl) at (rel axis cs:0,0);

            \nextgroupplot[
                title = {\textit{breverse-ldata}},
                % xmax = 10000,
                scaled x ticks = {base 10:-5},
                scaled y ticks = {base 10:-2},
                % xlabel = {number of compositions},
            ] 
                \qaddplot{breverse-ldata};
                \coordinate (botr) at (rel axis cs:1,0);
        \end{groupplot}

        \path (topl|-current bounding box.north)--
            coordinate(legendpos)
            (botr|-current bounding box.north);
        \matrix[
            matrix of nodes,
            anchor = south,
            draw,
            inner sep = 0.2em,
            draw
        ]at([yshift=1ex]legendpos) {
            \ref{plots:put} & $put$ & [2pt]
            \ref{plots:put$_m$} & $put_m$ & [2pt]
            \ref{plots:xpg} & $xpg$\\
        };

    \end{tikzpicture}

    \caption{Experimental results}
    \label{fig:result}

\end{figure}

Figure \ref{fig:result} 
% illustrates 
shows the evaluation times for each of the seven test cases using the three methods: $put$ in minBiGUL, $put_m$ in minBiGUL$_m$ and $xpg$.
% We simply ignore 
For simplicity, we omit the experimental results of $pg$, $cpg$ and $kpg$ because they are shown the slowness compared to $xpg$. For left associative compositions, $put$ works poorly because of the number of reevaluated $get$s.
%The figure shows $put_m$ and $xpg$ are better than $put$ in some cases.
The left part of the figure contains tests using not-large inputs, and we see that $put_m$ is the fastest method for them. However if the input size is large enough, in the cases of the right part of the figure, $put_m$ will be slower quickly due to time for manipulating data in the table. At that time, $xpg$ is 
% proved to be
the most effective method.
